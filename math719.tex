\documentclass[11pt]{article}

\usepackage{preamble}
%%%%% NEW MATH DEFINITIONS %%%%%

\usepackage{amsmath,amsfonts,bm}
% comment out the font for arxiv; iclr is times
% \usepackage{times}
\usepackage{amssymb}
\usepackage{amsthm}
\usepackage{graphicx}
\usepackage{caption}
\usepackage{color}
%\usepackage{media9}
%\usepackage{subfigure}
\usepackage{comment}
\usepackage{enumitem}
\usepackage{mathtools}
\usepackage{thmtools, thm-restate}
% \usepackage[ruled]{algorithm2e}
\usepackage{algorithm}
\usepackage[noend]{algorithmic}
\usepackage{verbatim}


% Custom Macros
\newcommand{\diag}{\operatorname{diag}}
\newcommand{\innp}[1]{\left\langle #1 \right\rangle}
\newcommand{\tildeF}{\widetilde{F}}
\newcommand{\tG}{\widetilde{G}}

\newcommand{\jd}[1]{{\color{purple}{\textbf{JD:} #1}}}
\newcommand{\xc}[1]{{\color{orange}{[\textbf{XC:} #1}]}}
\newcommand{\yw}[1]{{\color{teal}{[\textbf{AA:} #1}]}}
\newcommand{\yee}[1]{{\color{cyan}{[\textbf{YW:} #1}]}}

\newcommand{\DATE}[1]{{\color{purple}{[\textbf{Date:} #1}]}}



\iffalse
\theoremstyle{plain} \numberwithin{equation}{section}
\newtheorem{theorem}{Theorem}[section]
\numberwithin{theorem}{section}
\newtheorem{corollary}[theorem]{Corollary}
\newtheorem{conjecture}{Conjecture}
\newtheorem{lemma}[theorem]{Lemma}
\newtheorem{proposition}[theorem]{Proposition}
\newtheorem{claim}[theorem]{Claim}
\newtheorem{fact}[theorem]{Fact}
\theoremstyle{definition}
\newtheorem{definition}[theorem]{Definition}
\newtheorem{finalremark}[theorem]{Final Remark}
\newtheorem{remark}[theorem]{Remark}
\newtheorem{example}[theorem]{Example}
\newtheorem{observation}[theorem]{Observation}
\newtheorem{assumption}{Assumption}
\fi


% Mark sections of captions for referring to divisions of figures
\newcommand{\figleft}{{\em (Left)}}
\newcommand{\figcenter}{{\em (Center)}}
\newcommand{\figright}{{\em (Right)}}
\newcommand{\figtop}{{\em (Top)}}
\newcommand{\figbottom}{{\em (Bottom)}}
\newcommand{\captiona}{{\em (a)}}
\newcommand{\captionb}{{\em (b)}}
\newcommand{\captionc}{{\em (c)}}
\newcommand{\captiond}{{\em (d)}}

% Highlight a newly defined term
\newcommand{\newterm}[1]{{\bf #1}}


% Figure reference, lower-case.
\def\figref#1{figure~\ref{#1}}
% Figure reference, capital. For start of sentence
\def\Figref#1{Figure~\ref{#1}}
\def\twofigref#1#2{figures \ref{#1} and \ref{#2}}
\def\quadfigref#1#2#3#4{figures \ref{#1}, \ref{#2}, \ref{#3} and \ref{#4}}
% Section reference, lower-case.
\def\secref#1{section~\ref{#1}}
% Section reference, capital.
\def\Secref#1{Section~\ref{#1}}
% Reference to two sections.
\def\twosecrefs#1#2{sections \ref{#1} and \ref{#2}}
% Reference to three sections.
\def\secrefs#1#2#3{sections \ref{#1}, \ref{#2} and \ref{#3}}
% Reference to an equation, lower-case.
% \def\eqref#1{equation~\ref{#1}}
% Reference to an equation, upper case
% \def\Eqref#1{Equation~\ref{#1}}
% A raw reference to an equation---avoid using if possible
\def\plaineqref#1{\ref{#1}}
% Reference to a chapter, lower-case.
\def\chapref#1{chapter~\ref{#1}}
% Reference to an equation, upper case.
\def\Chapref#1{Chapter~\ref{#1}}
% Reference to a range of chapters
\def\rangechapref#1#2{chapters\ref{#1}--\ref{#2}}
% Reference to an algorithm, lower-case.
\def\algref#1{algorithm~\ref{#1}}
% Reference to an algorithm, upper case.
\def\Algref#1{Algorithm~\ref{#1}}
\def\twoalgref#1#2{algorithms \ref{#1} and \ref{#2}}
\def\Twoalgref#1#2{Algorithms \ref{#1} and \ref{#2}}
% Reference to a part, lower case
\def\partref#1{part~\ref{#1}}
% Reference to a part, upper case
\def\Partref#1{Part~\ref{#1}}
\def\twopartref#1#2{parts \ref{#1} and \ref{#2}}

\def\ceil#1{\lceil #1 \rceil}
\def\floor#1{\lfloor #1 \rfloor}
\def\1{\bm{1}}
\newcommand{\train}{\mathcal{D}}
\newcommand{\valid}{\mathcal{D_{\mathrm{valid}}}}
\newcommand{\test}{\mathcal{D_{\mathrm{test}}}}

\def\eps{{\epsilon}}


% Random variables
\def\reta{{\textnormal{$\eta$}}}
\def\ra{{\textnormal{a}}}
\def\rb{{\textnormal{b}}}
\def\rc{{\textnormal{c}}}
\def\rd{{\textnormal{d}}}
\def\re{{\textnormal{e}}}
\def\rf{{\textnormal{f}}}
\def\rg{{\textnormal{g}}}
\def\rh{{\textnormal{h}}}
\def\ri{{\textnormal{i}}}
\def\rj{{\textnormal{j}}}
\def\rk{{\textnormal{k}}}
\def\rl{{\textnormal{l}}}
% rm is already a command, just don't name any random variables m
\def\rn{{\textnormal{n}}}
\def\ro{{\textnormal{o}}}
\def\rp{{\textnormal{p}}}
\def\rq{{\textnormal{q}}}
\def\rr{{\textnormal{r}}}
\def\rs{{\textnormal{s}}}
\def\rt{{\textnormal{t}}}
\def\ru{{\textnormal{u}}}
\def\rv{{\textnormal{v}}}
\def\rw{{\textnormal{w}}}
\def\rx{{\textnormal{x}}}
\def\ry{{\textnormal{y}}}
\def\rz{{\textnormal{z}}}

% Random vectors
\def\rvepsilon{{\mathbf{\epsilon}}}
\def\rvtheta{{\mathbf{\theta}}}
\def\rva{{\mathbf{a}}}
\def\rvb{{\mathbf{b}}}
\def\rvc{{\mathbf{c}}}
\def\rvd{{\mathbf{d}}}
\def\rve{{\mathbf{e}}}
\def\rvf{{\mathbf{f}}}
\def\rvg{{\mathbf{g}}}
\def\rvh{{\mathbf{h}}}
\def\rvu{{\mathbf{i}}}
\def\rvj{{\mathbf{j}}}
\def\rvk{{\mathbf{k}}}
\def\rvl{{\mathbf{l}}}
\def\rvm{{\mathbf{m}}}
\def\rvn{{\mathbf{n}}}
\def\rvo{{\mathbf{o}}}
\def\rvp{{\mathbf{p}}}
\def\rvq{{\mathbf{q}}}
\def\rvr{{\mathbf{r}}}
\def\rvs{{\mathbf{s}}}
\def\rvt{{\mathbf{t}}}
\def\rvu{{\mathbf{u}}}
\def\rvv{{\mathbf{v}}}
\def\rvw{{\mathbf{w}}}
\def\rvx{{\mathbf{x}}}
\def\rvy{{\mathbf{y}}}
\def\rvz{{\mathbf{z}}}

% Elements of random vectors
\def\erva{{\textnormal{a}}}
\def\ervb{{\textnormal{b}}}
\def\ervc{{\textnormal{c}}}
\def\ervd{{\textnormal{d}}}
\def\erve{{\textnormal{e}}}
\def\ervf{{\textnormal{f}}}
\def\ervg{{\textnormal{g}}}
\def\ervh{{\textnormal{h}}}
\def\ervi{{\textnormal{i}}}
\def\ervj{{\textnormal{j}}}
\def\ervk{{\textnormal{k}}}
\def\ervl{{\textnormal{l}}}
\def\ervm{{\textnormal{m}}}
\def\ervn{{\textnormal{n}}}
\def\ervo{{\textnormal{o}}}
\def\ervp{{\textnormal{p}}}
\def\ervq{{\textnormal{q}}}
\def\ervr{{\textnormal{r}}}
\def\ervs{{\textnormal{s}}}
\def\ervt{{\textnormal{t}}}
\def\ervu{{\textnormal{u}}}
\def\ervv{{\textnormal{v}}}
\def\ervw{{\textnormal{w}}}
\def\ervx{{\textnormal{x}}}
\def\ervy{{\textnormal{y}}}
\def\ervz{{\textnormal{z}}}

% Random matrices
\def\rmA{{\mathbf{A}}}
\def\rmB{{\mathbf{B}}}
\def\rmC{{\mathbf{C}}}
\def\rmD{{\mathbf{D}}}
\def\rmE{{\mathbf{E}}}
\def\rmF{{\mathbf{F}}}
\def\rmG{{\mathbf{G}}}
\def\rmH{{\mathbf{H}}}
\def\rmI{{\mathbf{I}}}
\def\rmJ{{\mathbf{J}}}
\def\rmK{{\mathbf{K}}}
\def\rmL{{\mathbf{L}}}
\def\rmM{{\mathbf{M}}}
\def\rmN{{\mathbf{N}}}
\def\rmO{{\mathbf{O}}}
\def\rmP{{\mathbf{P}}}
\def\rmQ{{\mathbf{Q}}}
\def\rmR{{\mathbf{R}}}
\def\rmS{{\mathbf{S}}}
\def\rmT{{\mathbf{T}}}
\def\rmU{{\mathbf{U}}}
\def\rmV{{\mathbf{V}}}
\def\rmW{{\mathbf{W}}}
\def\rmX{{\mathbf{X}}}
\def\rmY{{\mathbf{Y}}}
\def\rmZ{{\mathbf{Z}}}

% Elements of random matrices
\def\ermA{{\textnormal{A}}}
\def\ermB{{\textnormal{B}}}
\def\ermC{{\textnormal{C}}}
\def\ermD{{\textnormal{D}}}
\def\ermE{{\textnormal{E}}}
\def\ermF{{\textnormal{F}}}
\def\ermG{{\textnormal{G}}}
\def\ermH{{\textnormal{H}}}
\def\ermI{{\textnormal{I}}}
\def\ermJ{{\textnormal{J}}}
\def\ermK{{\textnormal{K}}}
\def\ermL{{\textnormal{L}}}
\def\ermM{{\textnormal{M}}}
\def\ermN{{\textnormal{N}}}
\def\ermO{{\textnormal{O}}}
\def\ermP{{\textnormal{P}}}
\def\ermQ{{\textnormal{Q}}}
\def\ermR{{\textnormal{R}}}
\def\ermS{{\textnormal{S}}}
\def\ermT{{\textnormal{T}}}
\def\ermU{{\textnormal{U}}}
\def\ermV{{\textnormal{V}}}
\def\ermW{{\textnormal{W}}}
\def\ermX{{\textnormal{X}}}
\def\ermY{{\textnormal{Y}}}
\def\ermZ{{\textnormal{Z}}}

% Vectors
\def\vzero{{\bm{0}}}
\def\vone{{\bm{1}}}
\def\vmu{{\bm{\mu}}}
\def\vtheta{{\bm{\theta}}}
\def\va{{\bm{a}}}
\def\vb{{\bm{b}}}
\def\vc{{\bm{c}}}
\def\vd{{\bm{d}}}
% \def\ve{{\bm{e}}}
\def\ve{{\mathbf{e}}}
\def\vf{{\bm{f}}}
% \def\vg{{\bm{g}}}
\def\vg{{\mathbf{g}}}
\def\vh{{\bm{h}}}
\def\vi{{\bm{i}}}
\def\vj{{\bm{j}}}
\def\vk{{\bm{k}}}
\def\vl{{\bm{l}}}
\def\vm{{\bm{m}}}
\def\vn{{\bm{n}}}
\def\vo{{\bm{o}}}
\def\vp{{\bm{p}}}
\def\vq{{\bm{q}}}
\def\vr{{\bm{r}}}
\def\vs{{\bm{s}}}
\def\vt{{\bm{t}}}
% \def\vu{{\bm{u}}}
\def\vu{{\mathbf{u}}}
% \def\vv{{\bm{v}}}
\def\vv{{\mathbf{v}}}
% \def\vw{{\bm{w}}}
\def\vw{{\mathbf{w}}}
% \def\vx{{\bm{x}}}
\def\vx{{\mathbf{x}}}
% \def\vy{{\bm{y}}}
\def\vy{{\mathbf{y}}}
\def\vz{{\bm{z}}}
\def\dist{{\mathrm{dist}}}

% Elements of vectors
\def\evalpha{{\alpha}}
\def\evbeta{{\beta}}
\def\evepsilon{{\epsilon}}
\def\evlambda{{\lambda}}
\def\evomega{{\omega}}
\def\evmu{{\mu}}
\def\evpsi{{\psi}}
\def\evsigma{{\sigma}}
\def\evtheta{{\theta}}
\def\eva{{a}}
\def\evb{{b}}
\def\evc{{c}}
\def\evd{{d}}
\def\eve{{e}}
\def\evf{{f}}
\def\evg{{g}}
\def\evh{{h}}
\def\evi{{i}}
\def\evj{{j}}
\def\evk{{k}}
\def\evl{{l}}
\def\evm{{m}}
\def\evn{{n}}
\def\evo{{o}}
\def\evp{{p}}
\def\evq{{q}}
\def\evr{{r}}
\def\evs{{s}}
\def\evt{{t}}
\def\evu{{u}}
\def\evv{{v}}
\def\evw{{w}}
\def\evx{{x}}
\def\evy{{y}}
\def\evz{{z}}

% Matrix
\def\mA{{\bm{A}}}
\def\mB{{\bm{B}}}
\def\mC{{\bm{C}}}
\def\mD{{\bm{D}}}
\def\mE{{\bm{E}}}
\def\mF{{\bm{F}}}
\def\mG{{\bm{G}}}
\def\mH{{\bm{H}}}
\def\mI{{\bm{I}}}
\def\mJ{{\bm{J}}}
\def\mK{{\bm{K}}}
\def\mL{{\bm{L}}}
\def\mM{{\bm{M}}}
\def\mN{{\bm{N}}}
\def\mO{{\bm{O}}}
\def\mP{{\bm{P}}}
\def\mQ{{\bm{Q}}}
\def\mR{{\bm{R}}}
\def\mS{{\bm{S}}}
\def\mT{{\bm{T}}}
\def\mU{{\bm{U}}}
\def\mV{{\bm{V}}}
\def\mW{{\bm{W}}}
\def\mX{{\bm{X}}}
\def\mY{{\bm{Y}}}
\def\mZ{{\bm{Z}}}
\def\mBeta{{\bm{\beta}}}
\def\mPhi{{\bm{\Phi}}}
\def\mLambda{{\bm{\Lambda}}}
\def\mSigma{{\bm{\Sigma}}}

% Tensor
\DeclareMathAlphabet{\mathsfit}{\encodingdefault}{\sfdefault}{m}{sl}
\SetMathAlphabet{\mathsfit}{bold}{\encodingdefault}{\sfdefault}{bx}{n}
\newcommand{\tens}[1]{\bm{\mathsfit{#1}}}
\def\tA{{\tens{A}}}
\def\tB{{\tens{B}}}
\def\tC{{\tens{C}}}
\def\tD{{\tens{D}}}
\def\tE{{\tens{E}}}
\def\tF{{\tens{F}}}
\def\tG{{\tens{G}}}
\def\tH{{\tens{H}}}
\def\tI{{\tens{I}}}
\def\tJ{{\tens{J}}}
\def\tK{{\tens{K}}}
\def\tL{{\tens{L}}}
\def\tM{{\tens{M}}}
\def\tN{{\tens{N}}}
\def\tO{{\tens{O}}}
\def\tP{{\tens{P}}}
\def\tQ{{\tens{Q}}}
\def\tR{{\tens{R}}}
\def\tS{{\tens{S}}}
\def\tT{{\tens{T}}}
\def\tU{{\tens{U}}}
\def\tV{{\tens{V}}}
\def\tW{{\tens{W}}}
\def\tX{{\tens{X}}}
\def\tY{{\tens{Y}}}
\def\tZ{{\tens{Z}}}


% Graph
\def\gA{{\mathcal{A}}}
\def\gB{{\mathcal{B}}}
\def\gC{{\mathcal{C}}}
\def\gD{{\mathcal{D}}}
\def\gE{{\mathcal{E}}}
\def\gF{{\mathcal{F}}}
\def\gG{{\mathcal{G}}}
\def\gH{{\mathcal{H}}}
\def\gI{{\mathcal{I}}}
\def\gJ{{\mathcal{J}}}
\def\gK{{\mathcal{K}}}
\def\gL{{\mathcal{L}}}
\def\gM{{\mathcal{M}}}
\def\gN{{\mathcal{N}}}
\def\gO{{\mathcal{O}}}
\def\gP{{\mathcal{P}}}
\def\gQ{{\mathcal{Q}}}
\def\gR{{\mathcal{R}}}
\def\gS{{\mathcal{S}}}
\def\gT{{\mathcal{T}}}
\def\gU{{\mathcal{U}}}
\def\gV{{\mathcal{V}}}
\def\gW{{\mathcal{W}}}
\def\gX{{\mathcal{X}}}
\def\gY{{\mathcal{Y}}}
\def\gZ{{\mathcal{Z}}}


% Sets
\def\sA{{\mathbb{A}}}
\def\sB{{\mathbb{B}}}
\def\sC{{\mathbb{C}}}
\def\sD{{\mathbb{D}}}
% Don't use a set called E, because this would be the same as our symbol
% for expectation.
\def\sF{{\mathbb{F}}}
\def\sG{{\mathbb{G}}}
\def\sH{{\mathbb{H}}}
\def\sI{{\mathbb{I}}}
\def\sJ{{\mathbb{J}}}
\def\sK{{\mathbb{K}}}
\def\sL{{\mathbb{L}}}
\def\sM{{\mathbb{M}}}
\def\sN{{\mathbb{N}}}
\def\sO{{\mathbb{O}}}
\def\sP{{\mathbb{P}}}
\def\sQ{{\mathbb{Q}}}
\def\sR{{\mathbb{R}}}
\def\sS{{\mathbb{S}}}
\def\sT{{\mathbb{T}}}
\def\sU{{\mathbb{U}}}
\def\sV{{\mathbb{V}}}
\def\sW{{\mathbb{W}}}
\def\sX{{\mathbb{X}}}
\def\sY{{\mathbb{Y}}}
\def\sZ{{\mathbb{Z}}}

% Entries of a matrix
\def\emLambda{{\Lambda}}
\def\emA{{A}}
\def\emB{{B}}
\def\emC{{C}}
\def\emD{{D}}
\def\emE{{E}}
\def\emF{{F}}
\def\emG{{G}}
\def\emH{{H}}
\def\emI{{I}}
\def\emJ{{J}}
\def\emK{{K}}
\def\emL{{L}}
\def\emM{{M}}
\def\emN{{N}}
\def\emO{{O}}
\def\emP{{P}}
\def\emQ{{Q}}
\def\emR{{R}}
\def\emS{{S}}
\def\emT{{T}}
\def\emU{{U}}
\def\emV{{V}}
\def\emW{{W}}
\def\emX{{X}}
\def\emY{{Y}}
\def\emZ{{Z}}
\def\emSigma{{\Sigma}}

% entries of a tensor
% Same font as tensor, without \bm wrapper
\newcommand{\etens}[1]{\mathsfit{#1}}
\def\etLambda{{\etens{\Lambda}}}
\def\etA{{\etens{A}}}
\def\etB{{\etens{B}}}
\def\etC{{\etens{C}}}
\def\etD{{\etens{D}}}
\def\etE{{\etens{E}}}
\def\etF{{\etens{F}}}
\def\etG{{\etens{G}}}
\def\etH{{\etens{H}}}
\def\etI{{\etens{I}}}
\def\etJ{{\etens{J}}}
\def\etK{{\etens{K}}}
\def\etL{{\etens{L}}}
\def\etM{{\etens{M}}}
\def\etN{{\etens{N}}}
\def\etO{{\etens{O}}}
\def\etP{{\etens{P}}}
\def\etQ{{\etens{Q}}}
\def\etR{{\etens{R}}}
\def\etS{{\etens{S}}}
\def\etT{{\etens{T}}}
\def\etU{{\etens{U}}}
\def\etV{{\etens{V}}}
\def\etW{{\etens{W}}}
\def\etX{{\etens{X}}}
\def\etY{{\etens{Y}}}
\def\etZ{{\etens{Z}}}

%%% Distribution
% The true underlying data generating distribution
\newcommand{\pdata}{p_{\rm{data}}}

% The empirical distribution defined by the training set
\newcommand{\ptrain}{\hat{p}_{\rm{data}}}
\newcommand{\Ptrain}{\hat{P}_{\rm{data}}}

% The model distribution
\newcommand{\pmodel}{p_{\rm{model}}}
\newcommand{\Pmodel}{P_{\rm{model}}}
\newcommand{\ptildemodel}{\tilde{p}_{\rm{model}}}

% Stochastic autoencoder distributions
\newcommand{\pencode}{p_{\rm{encoder}}}
\newcommand{\pdecode}{p_{\rm{decoder}}}
\newcommand{\precons}{p_{\rm{reconstruct}}}

% Laplace distribution
\newcommand{\laplace}{\mathrm{Laplace}} 

% Specific letter fonts
\newcommand{\E}{\mathbb{E}}
\newcommand{\Ls}{\mathcal{L}}
\newcommand{\R}{\mathbb{R}}
\newcommand{\emp}{\tilde{p}}
\newcommand{\lr}{\alpha}
\newcommand{\reg}{\lambda}
\newcommand{\rect}{\mathrm{rectifier}}
\newcommand{\softmax}{\mathrm{softmax}}
\newcommand{\sigmoid}{\sigma}
\newcommand{\softplus}{\zeta}
\newcommand{\KL}{D_{\mathrm{KL}}}
\newcommand{\Var}{\mathrm{Var}}
\newcommand{\standarderror}{\mathrm{SE}}
\newcommand{\Cov}{\mathrm{Cov}}
\newcommand{\Rho}{\mathrm{P}}

% Wolfram Mathworld says $L^2$ is for function spaces and $\ell^2$ is for vectors
% But then they seem to use $L^2$ for vectors throughout the site, and so does
% wikipedia.
\newcommand{\normlzero}{L^0}
\newcommand{\normlone}{L^1}
\newcommand{\normltwo}{L^2}
\newcommand{\normlp}{L^p}
\newcommand{\normmax}{L^\infty}

\newcommand{\parents}{Pa} % See usage in notation.tex. Chosen to match Daphne's book.

% \DeclareMathOperator*{\argmax}{arg\,max}
% \DeclareMathOperator*{\argmin}{arg\,min}
\DeclareMathOperator*{\argmax}{argmax}
\DeclareMathOperator*{\argmin}{argmin}

\DeclareMathOperator{\sign}{sign}
\DeclareMathOperator{\Tr}{Tr}
\let\ab\allowbreak

\newcommand\independent{\protect\mathpalette{\protect\independenT}{\perp}}
\def\independenT#1#2{\mathrel{\rlap{$#1#2$}\mkern2mu{#1#2}}}


\let\underbrace\LaTeXunderbrace
\let\overbrace\LaTeXoverbrace

% INTEGRALS%%%%%%%%%%%%%%%%%%%%%%%%%%%%%%%% From Pascal
\def\Xint#1{\mathchoice
{\XXint\displaystyle\textstyle{#1}}%
{\XXint\textstyle\scriptstyle{#1}}%
{\XXint\scriptstyle\scriptscriptstyle{#1}}%
{\XXint\scriptscriptstyle%
\scriptscriptstyle{#1}}%
\!\int}
\def\XXint#1#2#3{{\setbox0=\hbox{$#1{#2#3}{%
\int}$ }
\vcenter{\hbox{$#2#3$ }}\kern-.6\wd0}}
\def\barint{\, \Xint -} % \, corrects the \! used in the definition
\def\bariint{\barint_{} \kern-.4em \barint}
\def\bariiint{\bariint_{} \kern-.4em \barint}
\renewcommand{\iint}{\int_{}\kern-.34em \int} %\ minor space between the integrals
\renewcommand{\iiint}{\iint_{}\kern-.34em \int} %\ minor space between the integrals

\title{Notes of Math 719: Partial Differential Equation\\Instructor: Dallas Albritton}
\author{YI WEI}
\date{Sep 2024}


\begin{document}

\maketitle
\tableofcontents

\section{Laplace's equation}
\DATE{Sep 4, 2024}

Elliptic PDEs:
\begin{example}
    Laplacian:
    \begin{align*}
        \Delta = \frac{\partial^2}{\partial x_1^2} + \dots +  \frac{\partial }{\partial x_n^2}\\
               = div \nabla
    \end{align*}
\end{example}

\begin{example}
    n = 3: Newtonian gravity \\
    \begin{align*}
        \text{magnitutde} &= \frac{\kappa m m_1}{\| a - a_1 \| } \quad \text{(inverse square)} \\
        m\ddot{a} &= \frac{\kappa m m_1}{\| a - a_1 \| }  \frac{a_1 - a}{\| a_1 - a \| } \\
            &= - m\nabla u(a)\\
        \text{where} \quad u(x) &= \frac{-k m_1}{\| x - a_1 \| }
    \end{align*}
\end{example}

N masses $m_1, \ldots , m_{N} \ge 0$ , location $a_1, \ldots , a_{N} \in \R^3$


$$
u(x) = -\kappa \sum_{k=1}^{N} \frac{m_k}{\| x - a_k \| }
$$


\begin{example}
    continuous distribution of mass
    \begin{align*}
        \varrho(x) \ge 0 \\
        u(x) = -\kappa \int_{\R^3} \frac{\varrho(y)}{\| x-y \| } \, dy
    \end{align*}
    suppose supp $\varrho \subseteq \omega$ bdd open set
\end{example}

\remark $supp \varrho := \overline{\{ x \in \R^3: S(x) \neq  0 \}} $

Laplace 
\begin{align*}
    \Delta u(x) = 0 \, for x \in \R^3  \bar{\omega}
\end{align*}


\begin{align*}
    \Delta_x u(x) = -\kappa \int_{\omega} \varrho(y) \Delta_x \frac{1}{\| x-y \| } \, dy
\end{align*}

\begin{align*}
    \Delta \frac{1}{\| x \| } = div \nabla \frac{1}{\| x \| }\\
                             = div(-\frac{x}{\| x \|^3 })\\
                             = \big(  -\frac{3}{\| x \|^3 } + 3x \frac{x^2}{\| x \|^5 }  \big)
\end{align*}

\begin{example}
    \begin{align*}
        \varrho = \text{const on } B_\R \\
            = \frac{m}{4 \pi \R^3} \\
        \text{and} u = -\kappa m \int_{B_\R} \frac{1}{\| x-y \| } \, dy
    \end{align*}
    \begin{claim}
        \begin{align*}
            u(x) = u(Ox) \, \text{where} O \in SO(3) \\
            u(Ox) = - \kappa m \int_{B_\R} \frac{1}{\| Ox - Oz \| } \, dz\\
                    = - \kappa m \int_{B_\R} \frac{1}{\| x - z \| } \, dz
        \end{align*}
    \end{claim}
    \begin{align*}
        u(x) = u(r) \\
        u: \R^3 \to \R \\
        u(x) = f(r)
    \end{align*}
\end{example}



\yee{TODO: Complete notes between Sep 4 and Sep 13}
\vspace{10mm}


\DATE{Sep 13, 2024}

Last time: Interior estimate

Suppose $R > 0$ and $u \in C^{2}(B_{2R})$ is harmonic, then
\begin{equation}
    \label{eq:1}
    \| \nabla^{k}u(x)  \| \le C_{k} \frac{1}{R^{k+3}} \int_{B_{2R}\setminus B_{R}}
    \| u \| \, dy \quad \forall  x \in B_{R}
\end{equation}

Question: Why estimate of this form?

Symmetries of $\Delta u = 0$. 

\begin{enumerate}
    \item homeoneity/scaler mult
    \begin{align*}
        u \mapsto \mu \quad (m \in \mathbb{R})
    \end{align*}
    \item scaling symmetry
    \begin{align*}
        u_{\lambda (x)} := u(\lambda x)\\
        u \mapsto u_{\lambda} \; (u(\lambda\cdot ))
    \end{align*}
    Here $\Delta u_{\lambda} = \Delta[u(\lambda x)] = \lambda^2 \Delta u (\lambda u)=0$
    \item translation:
    \begin{align*}
        u \mapsto u(\cdot -x_{0}) \quad (x_{0} \in \mathbb{R}^3)
    \end{align*}
    \item rotation and reflection
    \begin{align*}
        u \mapsto u(\mathcal{O}^{-1}) \quad (\mathcal{O} \in O(n))
    \end{align*}
\end{enumerate}

\begin{enumerate}
    \item $k=0, R=1$, What if
    \begin{align*}
        \|m u \| (x) \le C \int_{B_2 \setminus B_1} \| m u \| ^2 \, dy \quad 
        \text{in} \; B_1\\
        \Longrightarrow \| u(x) \| \le Cm \int_{B_2 \setminus B_1} \| u \| ^2 \, dy\\
        \Longrightarrow u = 0
    \end{align*}
    Then this is nonsense. We need the power in the integral to be 1, i.e., 
    \begin{align*}
        \|m u \| (x) \le C \int_{B_2 \setminus B_1} \| m u \|  \, dy \quad 
        \text{in} \; B_1
    \end{align*}
\end{enumerate}
To prove~\eqref{eq:1}, it suffices to prove it for
\begin{align*}
    \int _{B_{2R}\setminus B_{R}} \| u \| = 1
\end{align*}
Suppose~\eqref{eq:1} holds for all harmonic function $u$ s.t.
\begin{align*}
    \int _{B_{2R}\setminus B_{R}} \| u \| = 1
\end{align*}

Given arbitrary harmonic function $v$, we define
\begin{align*}
    u := \frac{v}{\int _{B_{2R}\setminus B_{R}}\| v \|  \, dy}
\end{align*}
Then $u$ is still harmonic
\begin{align*}
    \Longrightarrow \frac{\| v \| }{\int_{B_{2R}\setminus B_{R}} \| v \| \, dy}
    = \| u \|  \le C \frac{1}{R^{3}}
\end{align*}
To prove~\eqref{eq:1}, it suffices to prove it for $R=1$

Given harmonic function $v$ on $B_{2R}$
\begin{align*}
    u(x) := v(Rx)
\end{align*}
Here $u(x)$ is a harmonic function on $B_2$.

Then
\begin{align*}
    \| \nabla^{k} u(x) \|  &\le C \int_{B_2\setminus B_1} u(y) \, dy\\
    &=\int_{B_2 \setminus B_1} \| v(Ry) \| \, dy\\
    (\text{Change of Variable: } z=Ry) \\
    &= \frac{C}{R^{3}} \int_{B_{2R}\setminus B_{R}}\int \| v(z) \| \, dz
\end{align*}
where $\| \nabla^{k}(v(Rx)) \| := R^{k}\| (\nabla^k v)(Rx) \| = R^{k}\| 
(\nabla^{k}v)(q) \| $ where $q=Rx \in B_{R}$

To prove~\eqref{eq:1}, it's enough to do it for $R=1$ and 
$\int_{B_2 \setminus B_1} \| u \|\, dy = 1 $

\subsection{}
Given 
\begin{align}
    \label{eq:2}
    \phi u = 2 \Delta * (\Delta \phi u) + G * (\Delta \phi u)
\end{align}
where $u \in C^{2}$ and $\Delta u = 0$ and $\phi \in C_{0}^{\infty}$

$\underline{\text{One option}}$ $u \in L_{loc}^{1}$ ($u \in L^{1}(K)$) where K is compact
, it's called locally integrable, is harmonic if~\eqref{eq:2} holds $\forall  \phi \in C_{0}^{\infty}$

$\underline{\text{Another option}}$ $u \in L_{loc}^{1}$ is harmonic if
\begin{align*}
    \nabla (u * \phi_{\epsilon}) = 0 \qquad \text{for all } \epsilon
\end{align*}

\begin{example}
    \item $\frac{1}{|x|}$ is $L_{loc}^{1}(\mathbb{R}^3)$
    \item $(1+|x|)^{-3-\epsilon} \in L^{1}(\mathbb{R}^{3})$
\end{example}

\vspace{7mm}
\begin{definition}
    $\Omega \subseteq  R^{3}$ open, $u \in L_{loc}^{1}(\Omega)$ is weakly harmonic
    if 
    \begin{align*}
        \int u \nabla \phi \, dy = 0 \quad \forall \phi \in C_{0}^{\infty}(\Omega)
    \end{align*}
\end{definition}
\begin{remark}
    \begin{align*}
        \int \Delta u \cdot \phi = \int (div \nabla u) \phi\\
        = - \int \nabla u \nabla \phi = \int u \nabla \phi
    \end{align*}
    This is integration by parts.
\end{remark}

If you need a generalization, you need to make it easy to check and easy
to work with.

\begin{lemma}[Weyl's lemma]
    \, 

    If $u$ is weakly harmonic in $\Omega$, then $u$ is smooth and
    $\Delta u = 0$ in $\Omega$
\end{lemma}

\vspace{7mm}
To prove this, we need the following claim:
\begin{claim}
    \begin{enumerate}
        \item If $u$ is $C^{2}$ and $\Delta u = 0$, then $u$ is weakly harmonic
        \item If $u$ is $C^{2}$ and weakly harmonic, them $\Delta u =0$
    \end{enumerate}    
\end{claim}

\begin{proof}[Proof of Claim 1.3.]
    Suppose not.

    \begin{align*}
        \int u \Delta \phi = 0 \quad \forall \phi
    \end{align*}
    But $\exists x_{0} $ s.t. $\Delta u(x_{0}) \neq 0$
    \begin{align*}
        \int \Delta u \phi = 0 \quad \phi
    \end{align*}

    Choose $\phi$ s.t.
    \begin{align*}
        \int \Delta u \phi \neq 0
    \end{align*}
    Countradiction.
\end{proof}

\begin{proof}[proof of Wely's lemma]
    \,

    (3) $u \in L_{loc}^{1}$ is weakly harmonic, then $\phi_{\epsilon}*u$ is 
    also weakly harmonic.
    \begin{align*}
        \Longrightarrow \phi_{\epsilon} * u \text{ is strongly harmonic}
    \end{align*}
    We need to check
    \begin{align*}
        \int (u * \phi_{\epsilon}) \Delta \psi \, dy = 0 \quad \forall \psi
    \end{align*}
        \DATE{Sep 16, 2024}
    \begin{remark}
        $\Delta(f * \phi_{\epsilon}) = \Delta f * \phi_{\epsilon} \Longrightarrow \text{ Mollify harmonic function
        , get a harmonic function.}$
    \end{remark}

    Enough to work with balls. Enough to work in $B_{3}$ and prove smoothness in $B_{1}$.

    Because of translation and scalling symmetry.

    $u \in L^{1}(B_{3})$ weakly harmonic. We define
    \begin{align*}
        u_{\epsilon}(x) = u * \phi_{\epsilon}(x) \quad \text{for } x \in B_{2} \text{ and } 0 < \epsilon \le \frac{1}{2}
    \end{align*}
    Want to $\underline{\text{check}} \; u_{\epsilon}$ is weakly harmonic in $B_{2}$.

    \begin{align*}
        \forall \psi \in C_{0}^{\infty}(B_{2}): \; \int _{B_{2}} u_{\epsilon}(x)\Delta \psi(x) \, dx 
        = \int_{B_2}\int _{\mathbb{R}^3} \phi_{\epsilon}(x-y) u(y) \Delta \psi(x)\,dy\,dx\\
        = \int u(y)(\phi_{\epsilon}(-\cdot )*\Delta \psi)(y)\,dy\\
        = \int_{\mathbb{R}^{2}} u(y)\Delta(\phi_{\epsilon}(-\cdot )\psi)(y)\,dy\\
        = 0 \;(\text{by definition of weakly harmonic})
    \end{align*}
\end{proof}

Use interior estimates on $u_{\epsilon}$
\begin{align*}
    |\nabla^{k} u_{\epsilon}| \le C_{k}\int_{B_2\setminus B_1} \int |u_{\epsilon}|\,dy
    \le C_k\int _{B_3}|u|\,dy
\end{align*}

Use Arzela-Ascoli: For all $k$, $\nabla^{k}u_{\epsilon} \longrightarrow \nabla^{k}u$ uniformly in $B_1$

Weak version of $\Delta u = f$?
\begin{definition}
    $\Omega \subseteq \mathbb{R}^{3}$ open, $u,f \in L^{1}_{loc}(\Omega)$. Then we say that
    \begin{align*}
        \Delta u = f \quad \text{weakly harmonic in } \Omega
    \end{align*}
    if 
    \begin{align*}
        \int u \Delta \psi = \int f \psi \quad \forall \psi \in C_{0}^{\infty}(\Omega)
    \end{align*}
\end{definition}

Previously: $f \in C^{2}$ and compactly supported
\begin{align*}
    \Delta(G*f) = f
\end{align*}
General $f$ holds?

\begin{example}
    $G*f$ makes sense for $f \in L^{1}$

    \begin{align*}
        -\frac{1}{4\pi} \int \frac{f(y)}{|x-y|}\, dy
    \end{align*}
    Notice that $\frac{1}{|x|}$ is not integrable.
\end{example}

Now let $-\frac{1}{4\pi|x|}$ = G. And $G_1 = G\1_{B_1} \in L^{1} \bigcap L^{3-}$, 
$G_2 = G\1_{\mathbb{R}^{3}\setminus B_1} \in L^{\infty} \bigcap L^{3+}$, 
where $L^{3}$ means that $L^{3-\epsilon} \quad \forall \epsilon > 0$.
\begin{align*}
    \int_{B_1}\frac{1}{|x|^{3}} = c\int_{r=0}^{1} r^{-3}r^{2}  \mathrm{d}x = \infty\\
    \int_{\mathbb{R}^{3}\setminus B_1} = c \int_{r=1}^{\infty} r^{-3}r^{2}  \mathrm{d}x = \infty
\end{align*}


And $G*f =  \underbrace{G_1*f}_{\in L^{1} \bigcap L^{3-}}  + \underbrace{ G_2*f}_{\in
L^{3+}\bigcap L^{\infty}} \in L_{loc}^{1}$
\begin{exercise}
    $f \in L^{1} + L^{p}$ if $p < \frac{n}{2}$
\end{exercise}

\begin{claim}
    $\Delta(G*f) = f$ is weakly harmonic in $\mathbb{R}^{3}$
\end{claim}
\begin{proof}
    $\underline{\text{To check: }}$
    \begin{align*}
        \int(G*f)\Delta \varphi = \int f \varphi \forall \varphi \in C_{0}^{\infty}(\mathbb{R}^{3})\\
        \int(G*f)\Delta \varphi = \lim_{\epsilon \to 0^{+}}\int(K_{\epsilon}*f)\Delta \varphi = 
        \lim_{\epsilon \to 0^{+}} \int (\Delta K_{\epsilon}) * f \varphi\\
        = \int f \varphi
    \end{align*}
\end{proof}

\begin{proposition}
    Suppose $u_1, u_2$, $f \in L_{loc}^{1}$ and $\Delta u_1 = \Delta u_2 = f$ weakly harmonic in $\mathbb{R}^{3}$
    Then
    \begin{enumerate}
        \item $u_1 - u_2$ is smooth and harmonic 
        \item If $u_1,u_2$ are bounded, $u_1-u_2$ is constant.
        \item If $|u_1|, |u_2| \longrightarrow 0$ as $|x| \longrightarrow \infty$, (say $u_1,u_2 \in
        L^{1} + L^{p}, p < \infty$)
        then $u_1 \equiv u_2$.
        \begin{exercise}
            show $u_1 - u_2$ is bounded
        \end{exercise}
    \end{enumerate}
\end{proposition}

\end{document}