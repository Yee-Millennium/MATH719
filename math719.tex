\documentclass[11pt]{article}

\usepackage{preamble}
\input{math_commands.tex}

\title{Notes of Math 719: Partial Differential Equation\\Instructor: Dallas Albritton}
\author{YI WEI}
\date{Sep 2024}


\begin{document}

\maketitle
\tableofcontents

\section{Laplace's equation}
\DATE{Sep 4, 2024}

Elliptic PDEs:
\begin{example}
    Laplacian:
    \begin{align*}
        \Delta = \frac{\partial^2}{\partial x_1^2} + \dots +  \frac{\partial }{\partial x_n^2}\\
               = div \nabla
    \end{align*}
\end{example}

\begin{example}
    n = 3: Newtonian gravity \\
    \begin{align*}
        \text{magnitutde} &= \frac{\kappa m m_1}{\| a - a_1 \| } \quad \text{(inverse square)} \\
        m\ddot{a} &= \frac{\kappa m m_1}{\| a - a_1 \| }  \frac{a_1 - a}{\| a_1 - a \| } \\
            &= - m\nabla u(a)\\
        \text{where} \quad u(x) &= \frac{-k m_1}{\| x - a_1 \| }
    \end{align*}
\end{example}

N masses $m_1, \ldots , m_{N} \ge 0$ , location $a_1, \ldots , a_{N} \in \R^3$


$$
u(x) = -\kappa \sum_{k=1}^{N} \frac{m_k}{\| x - a_k \| }
$$


\begin{example}
    continuous distribution of mass
    \begin{align*}
        \varrho(x) \ge 0 \\
        u(x) = -\kappa \int_{\R^3} \frac{\varrho(y)}{\| x-y \| } \, dy
    \end{align*}
    suppose supp $\varrho \subseteq \omega$ bdd open set
\end{example}

\remark $supp \varrho := \overline{\{ x \in \R^3: S(x) \neq  0 \}} $

Laplace 
\begin{align*}
    \Delta u(x) = 0 \, for x \in \R^3  \bar{\omega}
\end{align*}


\begin{align*}
    \Delta_x u(x) = -\kappa \int_{\omega} \varrho(y) \Delta_x \frac{1}{\| x-y \| } \, dy
\end{align*}

\begin{align*}
    \Delta \frac{1}{\| x \| } = div \nabla \frac{1}{\| x \| }\\
                             = div(-\frac{x}{\| x \|^3 })\\
                             = \big(  -\frac{3}{\| x \|^3 } + 3x \frac{x^2}{\| x \|^5 }  \big)
\end{align*}

\begin{example}
    \begin{align*}
        \varrho = \text{const on } B_\R \\
            = \frac{m}{4 \pi \R^3} \\
        \text{and} u = -\kappa m \int_{B_\R} \frac{1}{\| x-y \| } \, dy
    \end{align*}
    \begin{claim}
        \begin{align*}
            u(x) = u(Ox) \, \text{where} O \in SO(3) \\
            u(Ox) = - \kappa m \int_{B_\R} \frac{1}{\| Ox - Oz \| } \, dz\\
                    = - \kappa m \int_{B_\R} \frac{1}{\| x - z \| } \, dz
        \end{align*}
    \end{claim}
    \begin{align*}
        u(x) = u(r) \\
        u: \R^3 \to \R \\
        u(x) = f(r)
    \end{align*}
\end{example}



\yee{TODO: Complete notes between Sep 4 and Sep 13}
\vspace{10mm}


\DATE{Sep 13, 2024}

Last time: Interior estimate

Suppose $R > 0$ and $u \in C^{2}(B_{2R})$ is harmonic, then
\begin{equation}
    \label{eq:1}
    \| \nabla^{k}u(x)  \| \le C_{k} \frac{1}{R^{k+3}} \int_{B_{2R}\setminus B_{R}}
    \| u \| \, dy \quad \forall  x \in B_{R}
\end{equation}

Question: Why estimate of this form?

Symmetries of $\Delta u = 0$. 

\begin{enumerate}
    \item homeoneity/scaler mult
    \begin{align*}
        u \mapsto \mu \quad (m \in \mathbb{R})
    \end{align*}
    \item scaling symmetry
    \begin{align*}
        u_{\lambda (x)} := u(\lambda x)\\
        u \mapsto u_{\lambda} \; (u(\lambda\cdot ))
    \end{align*}
    Here $\Delta u_{\lambda} = \Delta[u(\lambda x)] = \lambda^2 \Delta u (\lambda u)=0$
    \item translation:
    \begin{align*}
        u \mapsto u(\cdot -x_{0}) \quad (x_{0} \in \mathbb{R}^3)
    \end{align*}
    \item rotation and reflection
    \begin{align*}
        u \mapsto u(\mathcal{O}^{-1}) \quad (\mathcal{O} \in O(n))
    \end{align*}
\end{enumerate}

\begin{enumerate}
    \item $k=0, R=1$, What if
    \begin{align*}
        \|m u \| (x) \le C \int_{B_2 \setminus B_1} \| m u \| ^2 \, dy \quad 
        \text{in} \; B_1\\
        \Longrightarrow \| u(x) \| \le Cm \int_{B_2 \setminus B_1} \| u \| ^2 \, dy\\
        \Longrightarrow u = 0
    \end{align*}
    Then this is nonsense. We need the power in the integral to be 1, i.e., 
    \begin{align*}
        \|m u \| (x) \le C \int_{B_2 \setminus B_1} \| m u \|  \, dy \quad 
        \text{in} \; B_1
    \end{align*}
\end{enumerate}
To prove~\eqref{eq:1}, it suffices to prove it for
\begin{align*}
    \int _{B_{2R}\setminus B_{R}} \| u \| = 1
\end{align*}
Suppose~\eqref{eq:1} holds for all harmonic function $u$ s.t.
\begin{align*}
    \int _{B_{2R}\setminus B_{R}} \| u \| = 1
\end{align*}

Given arbitrary harmonic function $v$, we define
\begin{align*}
    u := \frac{v}{\int _{B_{2R}\setminus B_{R}}\| v \|  \, dy}
\end{align*}
Then $u$ is still harmonic
\begin{align*}
    \Longrightarrow \frac{\| v \| }{\int_{B_{2R}\setminus B_{R}} \| v \| \, dy}
    = \| u \|  \le C \frac{1}{R^{3}}
\end{align*}
To prove~\eqref{eq:1}, it suffices to prove it for $R=1$

Given harmonic function $v$ on $B_{2R}$
\begin{align*}
    u(x) := v(Rx)
\end{align*}
Here $u(x)$ is a harmonic function on $B_2$.

Then
\begin{align*}
    \| \nabla^{k} u(x) \|  &\le C \int_{B_2\setminus B_1} u(y) \, dy\\
    &=\int_{B_2 \setminus B_1} \| v(Ry) \| \, dy\\
    (\text{Change of Variable: } z=Ry) \\
    &= \frac{C}{R^{3}} \int_{B_{2R}\setminus B_{R}}\int \| v(z) \| \, dz
\end{align*}
where $\| \nabla^{k}(v(Rx)) \| := R^{k}\| (\nabla^k v)(Rx) \| = R^{k}\| 
(\nabla^{k}v)(q) \| $ where $q=Rx \in B_{R}$

To prove~\eqref{eq:1}, it's enough to do it for $R=1$ and 
$\int_{B_2 \setminus B_1} \| u \|\, dy = 1 $

\subsection{}
Given 
\begin{align}
    \label{eq:2}
    \phi u = 2 \Delta * (\Delta \phi u) + G * (\Delta \phi u)
\end{align}
where $u \in C^{2}$ and $\Delta u = 0$ and $\phi \in C_{0}^{\infty}$

$\underline{\text{One option}}$ $u \in L_{loc}^{1}$ ($u \in L^{1}(K)$) where K is compact
, it's called locally integrable, is harmonic if~\eqref{eq:2} holds $\forall  \phi \in C_{0}^{\infty}$

$\underline{\text{Another option}}$ $u \in L_{loc}^{1}$ is harmonic if
\begin{align*}
    \nabla (u * \phi_{\epsilon}) = 0 \qquad \text{for all } \epsilon
\end{align*}

\begin{example}
    \item $\frac{1}{|x|}$ is $L_{loc}^{1}(\mathbb{R}^3)$
    \item $(1+|x|)^{-3-\epsilon} \in L^{1}(\mathbb{R}^{3})$
\end{example}

\vspace{7mm}
\begin{definition}
    $\Omega \subseteq  R^{3}$ open, $u \in L_{loc}^{1}(\Omega)$ is weakly harmonic
    if 
    \begin{align*}
        \int u \nabla \phi \, dy = 0 \quad \forall \phi \in C_{0}^{\infty}(\Omega)
    \end{align*}
\end{definition}
\begin{remark}
    \begin{align*}
        \int \Delta u \cdot \phi = \int (div \nabla u) \phi\\
        = - \int \nabla u \nabla \phi = \int u \nabla \phi
    \end{align*}
    This is integration by parts.
\end{remark}

If you need a generalization, you need to make it easy to check and easy
to work with.

\begin{lemma}[Weyl's lemma]
    \, 

    If $u$ is weakly harmonic in $\Omega$, then $u$ is smooth and
    $\Delta u = 0$ in $\Omega$
\end{lemma}

\vspace{7mm}
To prove this, we need the following claim:
\begin{claim}
    \begin{enumerate}
        \item If $u$ is $C^{2}$ and $\Delta u = 0$, then $u$ is weakly harmonic
        \item If $u$ is $C^{2}$ and weakly harmonic, them $\Delta u =0$
    \end{enumerate}    
\end{claim}

\begin{proof}[Proof of Claim 1.3.]
    Suppose not.

    \begin{align*}
        \int u \Delta \phi = 0 \quad \forall \phi
    \end{align*}
    But $\exists x_{0} $ s.t. $\Delta u(x_{0}) \neq 0$
    \begin{align*}
        \int \Delta u \phi = 0 \quad \phi
    \end{align*}

    Choose $\phi$ s.t.
    \begin{align*}
        \int \Delta u \phi \neq 0
    \end{align*}
    Countradiction.
\end{proof}

\begin{proof}[proof of Wely's lemma]
    \,

    (3) $u \in L_{loc}^{1}$ is weakly harmonic, then $\phi_{\epsilon}*u$ is 
    also weakly harmonic.
    \begin{align*}
        \Longrightarrow \phi_{\epsilon} * u \text{ is strongly harmonic}
    \end{align*}
    We need to check
    \begin{align*}
        \int (u * \phi_{\epsilon}) \Delta \psi \, dy = 0 \quad \forall \psi
    \end{align*}
        \DATE{Sep 16, 2024}
    \begin{remark}
        $\Delta(f * \phi_{\epsilon}) = \Delta f * \phi_{\epsilon} \Longrightarrow \text{ Mollify harmonic function
        , get a harmonic function.}$
    \end{remark}

    Enough to work with balls. Enough to work in $B_{3}$ and prove smoothness in $B_{1}$.

    Because of translation and scalling symmetry.

    $u \in L^{1}(B_{3})$ weakly harmonic. We define
    \begin{align*}
        u_{\epsilon}(x) = u * \phi_{\epsilon}(x) \quad \text{for } x \in B_{2} \text{ and } 0 < \epsilon \le \frac{1}{2}
    \end{align*}
    Want to $\underline{\text{check}} \; u_{\epsilon}$ is weakly harmonic in $B_{2}$.

    \begin{align*}
        \forall \psi \in C_{0}^{\infty}(B_{2}): \; \int _{B_{2}} u_{\epsilon}(x)\Delta \psi(x) \, dx 
        = \int_{B_2}\int _{\mathbb{R}^3} \phi_{\epsilon}(x-y) u(y) \Delta \psi(x)\,dy\,dx\\
        = \int u(y)(\phi_{\epsilon}(-\cdot )*\Delta \psi)(y)\,dy\\
        = \int_{\mathbb{R}^{2}} u(y)\Delta(\phi_{\epsilon}(-\cdot )\psi)(y)\,dy\\
        = 0 \;(\text{by definition of weakly harmonic})
    \end{align*}
\end{proof}

Use interior estimates on $u_{\epsilon}$
\begin{align*}
    |\nabla^{k} u_{\epsilon}| \le C_{k}\int_{B_2\setminus B_1} \int |u_{\epsilon}|\,dy
    \le C_k\int _{B_3}|u|\,dy
\end{align*}

Use Arzela-Ascoli: For all $k$, $\nabla^{k}u_{\epsilon} \longrightarrow \nabla^{k}u$ uniformly in $B_1$

Weak version of $\Delta u = f$?
\begin{definition}
    $\Omega \subseteq \mathbb{R}^{3}$ open, $u,f \in L^{1}_{loc}(\Omega)$. Then we say that
    \begin{align*}
        \Delta u = f \quad \text{weakly harmonic in } \Omega
    \end{align*}
    if 
    \begin{align*}
        \int u \Delta \psi = \int f \psi \quad \forall \psi \in C_{0}^{\infty}(\Omega)
    \end{align*}
\end{definition}

Previously: $f \in C^{2}$ and compactly supported
\begin{align*}
    \Delta(G*f) = f
\end{align*}
General $f$ holds?

\begin{example}
    $G*f$ makes sense for $f \in L^{1}$

    \begin{align*}
        -\frac{1}{4\pi} \int \frac{f(y)}{|x-y|}\, dy
    \end{align*}
    Notice that $\frac{1}{|x|}$ is not integrable.
\end{example}

Now let $-\frac{1}{4\pi|x|}$ = G. And $G_1 = G\1_{B_1} \in L^{1} \bigcap L^{3-}$, 
$G_2 = G\1_{\mathbb{R}^{3}\setminus B_1} \in L^{\infty} \bigcap L^{3+}$, 
where $L^{3}$ means that $L^{3-\epsilon} \quad \forall \epsilon > 0$.
\begin{align*}
    \int_{B_1}\frac{1}{|x|^{3}} = c\int_{r=0}^{1} r^{-3}r^{2}  \mathrm{d}x = \infty\\
    \int_{\mathbb{R}^{3}\setminus B_1} = c \int_{r=1}^{\infty} r^{-3}r^{2}  \mathrm{d}x = \infty
\end{align*}


And $G*f =  \underbrace{G_1*f}_{\in L^{1} \bigcap L^{3-}}  + \underbrace{ G_2*f}_{\in
L^{3+}\bigcap L^{\infty}} \in L_{loc}^{1}$
\begin{exercise}
    $f \in L^{1} + L^{p}$ if $p < \frac{n}{2}$
\end{exercise}

\begin{claim}
    $\Delta(G*f) = f$ is weakly harmonic in $\mathbb{R}^{3}$
\end{claim}
\begin{proof}
    $\underline{\text{To check: }}$
    \begin{align*}
        \int(G*f)\Delta \varphi = \int f \varphi \forall \varphi \in C_{0}^{\infty}(\mathbb{R}^{3})\\
        \int(G*f)\Delta \varphi = \lim_{\epsilon \to 0^{+}}\int(K_{\epsilon}*f)\Delta \varphi = 
        \lim_{\epsilon \to 0^{+}} \int (\Delta K_{\epsilon}) * f \varphi\\
        = \int f \varphi
    \end{align*}
\end{proof}

\begin{proposition}
    Suppose $u_1, u_2$, $f \in L_{loc}^{1}$ and $\Delta u_1 = \Delta u_2 = f$ weakly harmonic in $\mathbb{R}^{3}$
    Then
    \begin{enumerate}
        \item $u_1 - u_2$ is smooth and harmonic 
        \item If $u_1,u_2$ are bounded, $u_1-u_2$ is constant.
        \item If $|u_1|, |u_2| \longrightarrow 0$ as $|x| \longrightarrow \infty$, (say $u_1,u_2 \in
        L^{1} + L^{p}, p < \infty$)
        then $u_1 \equiv u_2$.
        \begin{exercise}
            show $u_1 - u_2$ is bounded
        \end{exercise}
    \end{enumerate}
\end{proposition}

\end{document}