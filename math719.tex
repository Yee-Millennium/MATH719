\documentclass[11pt]{article}

\usepackage{preamble}
\input{math_commands.tex}

\title{Notes of Math 719: Partial Differential Equation\\Instructor: Dallas Albritton}
\author{YI WEI}
\date{Sep 2024}


\begin{document}

\maketitle
\tableofcontents

\section{Laplace's equation}
\DATE{Sep 4, 2024}

Elliptic PDEs:
\begin{example}
    Laplacian:
    \begin{align*}
        \Delta = \frac{\partial^2}{\partial x_1^2} + \dots +  \frac{\partial }{\partial x_n^2}\\
               = div \nabla
    \end{align*}
\end{example}

\begin{example}
    n = 3: Newtonian gravity \\
    \begin{align*}
        \text{magnitutde} &= \frac{\kappa m m_1}{\| a - a_1 \| } \quad \text{(inverse square)} \\
        m\ddot{a} &= \frac{\kappa m m_1}{\| a - a_1 \| }  \frac{a_1 - a}{\| a_1 - a \| } \\
            &= - m\nabla u(a)\\
        \text{where} \quad u(x) &= \frac{-k m_1}{\| x - a_1 \| }
    \end{align*}
\end{example}

N masses $m_1, \ldots , m_{N} \ge 0$ , location $a_1, \ldots , a_{N} \in \R^3$


$$
u(x) = -\kappa \sum_{k=1}^{N} \frac{m_k}{\| x - a_k \| }
$$


\begin{example}
    continuous distribution of mass
    \begin{align*}
        \varrho(x) \ge 0 \\
        u(x) = -\kappa \int_{\R^3} \frac{\varrho(y)}{\| x-y \| } \, dy
    \end{align*}
    suppose supp $\varrho \subseteq \omega$ bdd open set
\end{example}

\remark $supp \varrho := \overline{\{ x \in \R^3: S(x) \neq  0 \}} $

Laplace 
\begin{align*}
    \Delta u(x) = 0 \, for x \in \R^3  \bar{\omega}
\end{align*}


\begin{align*}
    \Delta_x u(x) = -\kappa \int_{\omega} \varrho(y) \Delta_x \frac{1}{\| x-y \| } \, dy
\end{align*}

\begin{align*}
    \Delta \frac{1}{\| x \| } = div \nabla \frac{1}{\| x \| }\\
                             = div(-\frac{x}{\| x \|^3 })\\
                             = \big(  -\frac{3}{\| x \|^3 } + 3x \frac{x^2}{\| x \|^5 }  \big)
\end{align*}

\begin{example}
    \begin{align*}
        \varrho = \text{const on } B_\R \\
            = \frac{m}{4 \pi \R^3} \\
        \text{and} u = -\kappa m \int_{B_\R} \frac{1}{\| x-y \| } \, dy
    \end{align*}
    \begin{claim}
        \begin{align*}
            u(x) = u(Ox) \, \text{where} O \in SO(3) \\
            u(Ox) = - \kappa m \int_{B_\R} \frac{1}{\| Ox - Oz \| } \, dz\\
                    = - \kappa m \int_{B_\R} \frac{1}{\| x - z \| } \, dz
        \end{align*}
    \end{claim}
    \begin{align*}
        u(x) = u(r) \\
        u: \R^3 \to \R \\
        u(x) = f(r)
    \end{align*}
\end{example}



\yee{TODO: Complete notes between Sep 4 and Sep 13}
\vspace{10mm}


\DATE{Sep 13, 2024}

Last time: Interior estimate

Suppose $R > 0$ and $u \in C^{2}(B_{2R})$ is harmonic, then
\begin{equation}
    \label{eq:1}
    \| \nabla^{k}u(x)  \| \le C_{k} \frac{1}{R^{k+3}} \int_{B_{2R}\setminus B_{R}}
    \| u \| \, dy \quad \forall  x \in B_{R}
\end{equation}

Question: Why estimate of this form?

Symmetries of $\Delta u = 0$. 

\begin{enumerate}
    \item homeoneity/scaler mult
    \begin{align*}
        u \mapsto \mu \quad (m \in \mathbb{R})
    \end{align*}
    \item scaling symmetry
    \begin{align*}
        u_{\lambda (x)} := u(\lambda x)\\
        u \mapsto u_{\lambda} \; (u(\lambda\cdot ))
    \end{align*}
    Here $\Delta u_{\lambda} = \Delta[u(\lambda x)] = \lambda^2 \Delta u (\lambda u)=0$
    \item translation:
    \begin{align*}
        u \mapsto u(\cdot -x_{0}) \quad (x_{0} \in \mathbb{R}^3)
    \end{align*}
    \item rotation and reflection
    \begin{align*}
        u \mapsto u(\mathcal{O}^{-1}) \quad (\mathcal{O} \in O(n))
    \end{align*}
\end{enumerate}

\begin{enumerate}
    \item $k=0, R=1$, What if
    \begin{align*}
        \|m u \| (x) \le C \int_{B_2 \setminus B_1} \| m u \| ^2 \, dy \quad 
        \text{in} \; B_1\\
        \Longrightarrow \| u(x) \| \le Cm \int_{B_2 \setminus B_1} \| u \| ^2 \, dy\\
        \Longrightarrow u = 0
    \end{align*}
    Then this is nonsense. We need the power in the integral to be 1, i.e., 
    \begin{align*}
        \|m u \| (x) \le C \int_{B_2 \setminus B_1} \| m u \|  \, dy \quad 
        \text{in} \; B_1
    \end{align*}
\end{enumerate}
To prove~\eqref{eq:1}, it suffices to prove it for
\begin{align*}
    \int _{B_{2R}\setminus B_{R}} \| u \| = 1
\end{align*}
Suppose~\eqref{eq:1} holds for all harmonic function $u$ s.t.
\begin{align*}
    \int _{B_{2R}\setminus B_{R}} \| u \| = 1
\end{align*}

Given arbitrary harmonic function $v$, we define
\begin{align*}
    u := \frac{v}{\int _{B_{2R}\setminus B_{R}}\| v \|  \, dy}
\end{align*}
Then $u$ is still harmonic
\begin{align*}
    \Longrightarrow \frac{\| v \| }{\int_{B_{2R}\setminus B_{R}} \| v \| \, dy}
    = \| u \|  \le C \frac{1}{R^{3}}
\end{align*}
To prove~\eqref{eq:1}, it suffices to prove it for $R=1$

Given harmonic function $v$ on $B_{2R}$
\begin{align*}
    u(x) := v(Rx)
\end{align*}
Here $u(x)$ is a harmonic function on $B_2$.

Then
\begin{align*}
    \| \nabla^{k} u(x) \|  &\le C \int_{B_2\setminus B_1} u(y) \, dy\\
    &=\int_{B_2 \setminus B_1} \| v(Ry) \| \, dy\\
    (\text{Change of Variable: } z=Ry) \\
    &= \frac{C}{R^{3}} \int_{B_{2R}\setminus B_{R}}\int \| v(z) \| \, dz
\end{align*}
where $\| \nabla^{k}(v(Rx)) \| := R^{k}\| (\nabla^k v)(Rx) \| = R^{k}\| 
(\nabla^{k}v)(q) \| $ where $q=Rx \in B_{R}$

To prove~\eqref{eq:1}, it's enough to do it for $R=1$ and 
$\int_{B_2 \setminus B_1} \| u \|\, dy = 1 $

\subsection{}
Given 
\begin{align}
    \label{eq:2}
    \phi u = 2 \Delta * (\Delta \phi u) + G * (\Delta \phi u)
\end{align}
where $u \in C^{2}$ and $\Delta u = 0$ and $\phi \in C_{0}^{\infty}$

$\underline{\text{One option}}$ $u \in L_{loc}^{1}$ ($u \in L^{1}(K)$) where K is compact
, it's called locally integrable, is harmonic if~\eqref{eq:2} holds $\forall  \phi \in C_{0}^{\infty}$

$\underline{\text{Another option}}$ $u \in L_{loc}^{1}$ is harmonic if
\begin{align*}
    \nabla (u * \phi_{\epsilon}) = 0 \qquad \text{for all } \epsilon
\end{align*}

\begin{example}
    \item $\frac{1}{|x|}$ is $L_{loc}^{1}(\mathbb{R}^3)$
    \item $(1+|x|)^{-3-\epsilon} \in L^{1}(\mathbb{R}^{3})$
\end{example}

\vspace{7mm}
\begin{definition}
    $\Omega \subseteq  R^{3}$ open, $u \in L_{loc}^{1}(\Omega)$ is weakly harmonic
    if 
    \begin{align*}
        \int u \nabla \phi \, dy = 0 \quad \forall \phi \in C_{0}^{\infty}(\Omega)
    \end{align*}
\end{definition}
\begin{remark}
    \begin{align*}
        \int \Delta u \cdot \phi = \int (div \nabla u) \phi\\
        = - \int \nabla u \nabla \phi = \int u \nabla \phi
    \end{align*}
    This is integration by parts.
\end{remark}

If you need a generalization, you need to make it easy to check and easy
to work with.

\begin{lemma}[Weyl's lemma]
    \, 

    If $u$ is weakly harmonic in $\Omega$, then $u$ is smooth and
    $\Delta u = 0$ in $\Omega$
\end{lemma}

\vspace{7mm}
To prove this, we need the following claim:
\begin{claim}
    \begin{enumerate}
        \item If $u$ is $C^{2}$ and $\Delta u = 0$, then $u$ is weakly harmonic
        \item If $u$ is $C^{2}$ and weakly harmonic, them $\Delta u =0$
    \end{enumerate}    
\end{claim}

\begin{proof}[Proof of Claim 1.3.]
    Suppose not.

    \begin{align*}
        \int u \Delta \phi = 0 \quad \forall \phi
    \end{align*}
    But $\exists x_{0} $ s.t. $\Delta u(x_{0}) \neq 0$
    \begin{align*}
        \int \Delta u \phi = 0 \quad \phi
    \end{align*}

    Choose $\phi$ s.t.
    \begin{align*}
        \int \Delta u \phi \neq 0
    \end{align*}
    Countradiction.
\end{proof}

\begin{proof}[proof of Wely's lemma]
    \,

    (3) $u \in L_{loc}^{1}$ is weakly harmonic, then $\phi_{\epsilon}*u$ is 
    also weakly harmonic.
    \begin{align*}
        \Longrightarrow \phi_{\epsilon} * u \text{ is strongly harmonic}
    \end{align*}
    We need to check
    \begin{align*}
        \int (u * \phi_{\epsilon}) \Delta \psi \, dy = 0 \quad \forall \psi
    \end{align*}
        \DATE{Sep 16, 2024}
    \begin{remark}
        $\Delta(f * \phi_{\epsilon}) = \Delta f * \phi_{\epsilon} \Longrightarrow \text{ Mollify harmonic function
        , get a harmonic function.}$
    \end{remark}

    Enough to work with balls. Enough to work in $B_{3}$ and prove smoothness in $B_{1}$.

    Because of translation and scalling symmetry.

    $u \in L^{1}(B_{3})$ weakly harmonic. We define
    \begin{align*}
        u_{\epsilon}(x) = u * \phi_{\epsilon}(x) \quad \text{for } x \in B_{2} \text{ and } 0 < \epsilon \le \frac{1}{2}
    \end{align*}
    Want to $\underline{\text{check}} \; u_{\epsilon}$ is weakly harmonic in $B_{2}$.

    \begin{align*}
        \forall \psi \in C_{0}^{\infty}(B_{2}): \; \int _{B_{2}} u_{\epsilon}(x)\Delta \psi(x) \, dx 
        = \int_{B_2}\int _{\mathbb{R}^3} \phi_{\epsilon}(x-y) u(y) \Delta \psi(x)\,dy\,dx\\
        = \int u(y)(\phi_{\epsilon}(-\cdot )*\Delta \psi)(y)\,dy\\
        = \int_{\mathbb{R}^{2}} u(y)\Delta(\phi_{\epsilon}(-\cdot )\psi)(y)\,dy\\
        = 0 \;(\text{by definition of weakly harmonic})
    \end{align*}
\end{proof}

Use interior estimates on $u_{\epsilon}$
\begin{align*}
    |\nabla^{k} u_{\epsilon}| \le C_{k}\int_{B_2\setminus B_1} \int |u_{\epsilon}|\,dy
    \le C_k\int _{B_3}|u|\,dy
\end{align*}

Use Arzela-Ascoli: For all $k$, $\nabla^{k}u_{\epsilon} \longrightarrow \nabla^{k}u$ uniformly in $B_1$

Weak version of $\Delta u = f$?
\begin{definition}
    $\Omega \subseteq \mathbb{R}^{3}$ open, $u,f \in L^{1}_{loc}(\Omega)$. Then we say that
    \begin{align*}
        \Delta u = f \quad \text{weakly harmonic in } \Omega
    \end{align*}
    if 
    \begin{align*}
        \int u \Delta \psi = \int f \psi \quad \forall \psi \in C_{0}^{\infty}(\Omega)
    \end{align*}
\end{definition}

Previously: $f \in C^{2}$ and compactly supported
\begin{align*}
    \Delta(G*f) = f
\end{align*}
General $f$ holds?

\begin{example}
    $G*f$ makes sense for $f \in L^{1}$

    \begin{align*}
        -\frac{1}{4\pi} \int \frac{f(y)}{|x-y|}\, dy
    \end{align*}
    Notice that $\frac{1}{|x|}$ is not integrable.
\end{example}

Now let $-\frac{1}{4\pi|x|}$ = G. And $G_1 = G\1_{B_1} \in L^{1} \bigcap L^{3-}$, 
$G_2 = G\1_{\mathbb{R}^{3}\setminus B_1} \in L^{\infty} \bigcap L^{3+}$, 
where $L^{3}$ means that $L^{3-\epsilon} \quad \forall \epsilon > 0$.
\begin{align*}
    \int_{B_1}\frac{1}{|x|^{3}} = c\int_{r=0}^{1} r^{-3}r^{2}  \mathrm{d}x = \infty\\
    \int_{\mathbb{R}^{3}\setminus B_1} = c \int_{r=1}^{\infty} r^{-3}r^{2}  \mathrm{d}x = \infty
\end{align*}


And $G*f =  \underbrace{G_1*f}_{\in L^{1} \bigcap L^{3-}}  + \underbrace{ G_2*f}_{\in
L^{3+}\bigcap L^{\infty}} \in L_{loc}^{1}$
\begin{exercise}
    $f \in L^{1} + L^{p}$ if $p < \frac{n}{2}$
\end{exercise}

\begin{claim}
    $\Delta(G*f) = f$ is weakly harmonic in $\mathbb{R}^{3}$
\end{claim}
\begin{proof}
    $\underline{\text{To check: }}$
    \begin{align*}
        \int(G*f)\Delta \varphi = \int f \varphi \forall \varphi \in C_{0}^{\infty}(\mathbb{R}^{3})\\
        \int(G*f)\Delta \varphi = \lim_{\epsilon \to 0^{+}}\int(K_{\epsilon}*f)\Delta \varphi = 
        \lim_{\epsilon \to 0^{+}} \int (\Delta K_{\epsilon}) * f \varphi\\
        = \int f \varphi
    \end{align*}
\end{proof}

\begin{proposition}
    Suppose $u_1, u_2$, $f \in L_{loc}^{1}$ and $\Delta u_1 = \Delta u_2 = f$ weakly harmonic in $\mathbb{R}^{3}$
    Then
    \begin{enumerate}
        \item $u_1 - u_2$ is smooth and harmonic 
        \item If $u_1,u_2$ are bounded, $u_1-u_2$ is constant.
        \item If $|u_1|, |u_2| \longrightarrow 0$ as $|x| \longrightarrow \infty$, (say $u_1,u_2 \in
        L^{1} + L^{p}, p < \infty$)
        then $u_1 \equiv u_2$.
        \begin{exercise}
            show $u_1 - u_2$ is bounded
        \end{exercise}
    \end{enumerate}
\end{proposition}


\DATE{Sep 18, 2024}
\,

$\underline{\text{Mean value formula/property}}$
\begin{proposition}
    Suppose $u$ is harmonic on $B_{\mathbb{R}}(x_0)$. Then
    \begin{enumerate}
        \item $u(x_0) = \barint_{\partial B_{r}(x_0)} u(y)\, dS \quad \forall r \in (0,R)$
        $u(x_0) = \text{avg of }u \text{ over } \partial B_{r}(x_0)$
        \item $u(x_0) = \barint_{B_{r}(x_0)} = u(y)\, dy \quad \forall r \in (0,R)$
        $u(x_0) = \text{ avg of }u \text{ over } B_{r}(x_0)$
    \end{enumerate}
\end{proposition}
\begin{proof}
    \begin{enumerate}
        \item proof of (1).
        \begin{align*}
            \barint_{\partial B_{r}(x_0)} u(y)\, dS = q(r) \longrightarrow u(x_0) \text{ as }
                r \rightarrow 0^{+}
        \end{align*}
        To show: $\frac{dq}{dr}=0 \quad (q \equiv \text{ const })$ 
        \begin{align*}
            x_0 &= 0\\
            &= \frac{1}{|\partial B_1|r^{n-1}} \int _{\partial B_r}u(y)\, dS\\
            (r= rz \in \partial B_1) &= \frac{1}{|\partial B_1|} \int_{\partial B_1}u(rz)\, d\tilde{S}
        \end{align*}
        We know $dS = r^{n-1}d\tilde{S}$.
        \begin{align*}
            \frac{dq}{dr} = \frac{1}{|\partial B_1|} \int_{\partial B_1}(\nabla u)(rz)\cdot z \,d\tilde{S}\\
            &= -\frac{1}{|\partial B_1|}\int_{B_1}div[(\nabla u)(rz)]\, dz = 0\\
            \Longrightarrow \Delta u(rz) r = 0
        \end{align*}
        
        The second proof:

        $\varphi = G * (\Delta \varphi u + 2 \nabla \varphi \nabla u) := G*f$ where
        $f := \Delta \varphi u + 2 \nabla \varphi \nabla u$

        Choose $\varphi \equiv 1$ on $B_{r+\epsilon}$.

        \begin{align*}
            u(0) = (G*f)(0)\\
            &= \int G(0-y)f(y)\, dy
        \end{align*}
        \begin{align*}
            \barint _{B_r} u(x)\, dx = \barint_{B_r} \int_{y}G(x-y)f(y) \, dy dx\\
            = \int_{y} \barint_{B_r}G(x-y)\,dx f(y)\,dy
        \end{align*}
        Here we know $G(0-y)$ gravitational point of point mass at $0$ measured at $-y$.
        $\barint_{B_r}G(x-y)\, dx$ gravitational point of body with mass distributed over $B_r$ measured 
        at $-y$. 

    \end{enumerate}
\end{proof}

\subsection{Maximum principle}
\begin{lemma}
    Suppose $u$ is harmonic in $B_{R}(x_0)$ and suppose $u(x_0) = \sup_{B_{R}(x_0)}u(x)$, then
    \begin{align*}
        u \equiv \text{ const }
    \end{align*}
\end{lemma}
\begin{example}
    n = 2. 
    \begin{enumerate}
        \item $1,\, x_1,\,x_2$
        \item $x_1^{2} - x_2^{2}$. Suppose $u(0) = 0$, $\nabla u(0) = 0$
        \begin{align*}
            \nabla^{2} = 
            \begin{bmatrix}
                ,2\\
                1,2
            \end{bmatrix}
            =[]
        \end{align*}
        row avB $u = \frac{\lambda_1}{2}y_1^{2} + \frac{\lambda_2}{2}y_2^{2}$ and 
        $\text{tr}\nabla^{2}u = \Delta u = \lambda_1 + \lambda_2$
    \end{enumerate}
\end{example}
\begin{proof}
    \begin{align*}
        u(x_0) \ge u(x) \quad \forall  x \in b_{R}(x_0)\\
        u(x_0) \ge \barint_{B_{R}(x_0)}u(x)\, dx = u(x_0) \quad \text{ mean val property}
    \end{align*}
\end{proof}

\begin{remark}
    $\sup_{\Omega} u = \sup_{\partial \Omega} u $
\end{remark}
\begin{corollary}[strong maximum principle]
    $\Omega \in \mathbb{R}^{3}$ bounded domain

    Suppose $u$ harmonic in $\Omega$ and $u(x_0) = \sup_{\Omega}u \quad $ for some $x_0 \in \Omega$

    Then 
    \begin{align*}
        u \equiv \text{ const }
    \end{align*} 
\end{corollary}
\begin{remark}
    $S = \{ x \in \Omega: u(x) = u(x_0) \}$
\end{remark}
If $S$ is non-empty, open and closed (in $\Omega$). "Relatively closed".
Since $\Omega$ is connected:
\begin{align*}
    S = \Omega
\end{align*}
\begin{remark}
    When you have a maximum principle, and then you have a Minimum principle. And you have a
    Comparison principle
    \begin{align*}
        u_1,u_2 \text{ on } \Omega\\
        u_1 \ge u_2 \text{ on } \partial \Omega
    \end{align*}
    Then we have
    \begin{align*}
        u_1 > u_2 \text{ on } \Omega (\text{ or } u_1 = u_2)
    \end{align*}
    And we prove this by $u_1 - u_2$ also harmonic function. $u_1 - u_2 \ge 0$ on $\partial \Omega$
    . Then we apply min principle.
\end{remark}

\DATE{Sep 20, 2024}

Minimum Principle:

Given $\Omega \subseteq \mathbb{R}^{3}$ bounded domain. Suppose $u$ is harmonic in $\Omega$ and 
$u(x_0) = \inf_{\Omega}u$ for some $x_0 \in \Omega$. Then $u \equiv \text{ const}$.

\begin{proposition}[Harneck's principle/inequality]
    Suppose $u$ is harmonic and $u >0$ in $\Omega$. 
    Let $K \subseteq \Omega$ compact set. Then $\exists  C = C(\Omega,K) > 0$ s.t.
    \begin{align*}
        \sup_{K} u \le C\inf_{K}u
    \end{align*}
\end{proposition}
\begin{remark}
    This implies minimum principle: let $v = u-\inf_{K}u + \epsilon$

    We have $\inf_{K}v = \epsilon \Longrightarrow \sup_{K}v \le C\inf_{K}v = C\epsilon$
\end{remark}
\begin{proof}
    (In Evans, he proved with mean value formula. $u(x) = \barint_{B_{R}(x)}u\,dy$. 
    Relate the value between two balls.)

    We used to use scaling argument to prove things. Now we use compactness argument.

    Harneck: $\forall \Omega, \forall K \subseteq \Omega \text{ compact, }\exists C$ s.t.

    $\forall u > 0$ on $\Omega$ harmonic,
    \begin{align*}
        \sup_{K}u \le C\inf_{K}u
    \end{align*}

    Negsta: $\exists \Omega, \, \exists K \subseteq \Omega$ compact s.t. $\forall C > 0$:

    $\exists u_{C} > 0$ harmonic on $\Omega$ s.t.
    \begin{align*}
        \sup_{K}u > C\inf_{K}u
    \end{align*}

    Suppose $C = N \in \mathbb{N}$:
    \begin{align*}
        \sup_{K}u > C\inf_{K}u
    \end{align*}

    Define $v_{N} := \frac{u_{N}}{\sup_{K}u_{N}}$. We have
    \begin{align*}
        1 = \sup_{K}v_{N} > N\inf_{K}v_{N} \Longrightarrow \inf_{K}v_{N} < \frac{1}{N} 
    \end{align*}
    Let $a \in K$. Find $B_{r_{a}} \in \Omega$
    \begin{align*}
        1 \ge v_{N}(a) = \barint_{B_{r}(a)}v_{N}(y)\,dy
    \end{align*}
    \begin{align*}
        \underbrace{|\nabla v_{N}^{k}(x)|}_{x \in B_{\frac{r_a}{2}}(a)} 
            \le C(k)r_{a}^{-k}\barint_{B_{r}(a)}v_{N}(y)\,dy
    \end{align*}
    Interior estimates theorem: $\Omega_{1} = \bigcup_{i=1}^{M}B_{r_{a_1}}(a_i)$ finitely many balls.
    Here $\Omega_1$ covers $K$.
    \begin{align*}
        |\nabla^{k}v_{N}| \le C(k) \quad x \in \Omega_1
    \end{align*}
    And we have $C(k) = C \max_{i=1, \ldots ,M}r_{a_1}^{-k}$.

    By Arzela-Ascoli $\Longrightarrow \exists $subsequence (not relabeling) $v_{N} \to v_{\infty}$ 
    uniformly in $\Omega$, $v_{\infty}$ is harmonic in $\Omega$
    \begin{align*}
        \sup_{K}v_{\infty} = 1,\, \inf_{K} v_{\infty}=0,\, v_{\infty}\ge 0
    \end{align*}
    Countradiction! (Since this implies that $v$ is constant.)
\end{proof}
\begin{remark}
    It's called compactness-contradiction argument cf.Terry Tao.
\end{remark}

\subsection{Boundary value problems (BVPs)}
\begin{equation}
    \begin{cases} 
        \Delta u = f & \text{ in }\Omega \\ 
        u|_{\partial \Omega} = g  \quad \text{Dirichlet BCs} 
    \end{cases}
\end{equation}

\begin{example}
    electrostatics: density of temperature or concentration of chemical... We have $\underline{\text{diffusion}}$.
    
    Given $c(x,t)$. (Fick's law)
    \begin{align*}
        \frac{d}{dt} \int_{O}c(x,t)\,dx  &= \kappa \int_{\partial O}\nabla c \cdot \vec{n} dS\\
        &= \kappa \int_{O}div \nabla c \, dx = \kappa \int_{O} \Delta c \, dx \quad \forall O
    \end{align*}
    This is Heat equation 
    \begin{align*}
        \partial_{t}c = \kappa \Delta c \quad in \Omega
    \end{align*}
    If there is no flux:
    \begin{align*}
        \nabla c \cdot \vec{n}|_{\partial \Omega} = 0 \quad \text{Neumann BCs}\\
        \frac{\partial c}{\partial n} = 0
    \end{align*}
\end{example}

\DATE{Sep 23, 2024}

Def: $\mathbb{R}_{+}^{n} = \{ x \in \mathbb{R}^{n}: x_{n} > 0 \}$
Dirichlet problem and Neumann problem. 
\begin{equation}
    \begin{cases} 
    \Delta u = f, & \text{in} \Omega  \\ 
    u|_{\partial \Omega} = g   
    \end{cases}
\end{equation}

Now consider 

\begin{equation}
    \begin{cases} 
    \Delta u = f, & \text{in} \;\mathbb{R}_{+}^{3} \\ 
    u|_{\partial \mathbb{R}_{+}^{3}} = g   
    \end{cases}
\end{equation}

$R \vec{x} = (x_1,x_2,-x_3)$ and $\Delta[u(R\vec{x})] = (\Delta u)(R \vec{x})$

By Odd-in-$x_3$, we mean 
\begin{align*}
    u(\vec{x}) = -u(R\vec{x})\\
    u(x_1,x_2,x_3) = -u(x_1,x_2,-x_3)
\end{align*}
(If constant, then $u|_{\partial \mathbb{R}_{+}^{3}} = 0$)

\begin{equation}
    \tilde{f}(x) =
    \begin{cases} 
    f(\vec{x}), & x_3 > 0  \\ 
    -f(R\vec{x}), &    x_3 < 0
    \end{cases}
\end{equation}

\begin{equation}
    \tilde(u)(x) \text{ sol of } \Delta \tilde{u} = \tilde{u} = \tilde{f} \text{ on }\mathbb{R}^{3}
\end{equation}

\begin{equation}
    \tilde(u)(x) = (G * \tilde{f})(x)
\end{equation}
where $\tilde{u}$ is odd.

\begin{align*}
    \Delta \tilde{u} = \tilde{f}\\
    \underbrace{(\Delta \tilde{u}(R\vec{x}))}_{= \Delta[\tilde{u}(R\vec{x})]}  &= \tilde{f}(R\vec{x})\\
    &= - \tilde{f}(x)\\
    \Longrightarrow \Delta[-\tilde{u}(R\vec{x})] = \tilde{f}(x)
\end{align*}

sol unique means:
\begin{align*}
    \Longrightarrow \tilde(u)(\vec{x}) = -\tilde{u}(R\vec{x})\\
    \Longrightarrow \tilde{u} \text{ is odd}
\end{align*}

Now 
\begin{align*}
    \tilde{u}(x) = G*\tilde{f}\\
    = - \frac{1}{4\pi} \int_{\mathbb{R}^{3}} \frac{\tilde{f}(y)}{|x-y|}\,dy\\
    = -\frac{1}{4\pi}\int_{\mathbb{R}^{3}} \frac{1}{|x-y|}
     \Big[\1 _{\{ y_3 > 0 \}}f + \1_{\{ y_3 < 0 \}}(-f)(R\vec{y})\Big]\,dy\\
    = -\frac{1}{4\pi} \int_{\mathbb{R}_{+}^{3}} \frac{1}{|x-y|}f(y) - \frac{1}{|x-R\vec{y}|}f(y)\,dy\\
    \overset{y*=Ry}{=} - \frac{1}{4\pi} \int_{\mathbb{R}_{+}^{3}} \Big[\frac{1}{|x-y|}-\frac{1}{|x-y*|} \Big]f(y)\,dy \\
\end{align*}

Then we get:
\begin{align*}
    G_{\mathbb{R}_{+}^{3}}(x,y) = -\frac{1}{4\pi} (\frac{1}{|x-y|} - \frac{1}{|x-y*|}) \quad
        \text{ Green's function for half space}
\end{align*}

Thus the Boundary value problem is:
\begin{equation}
    \begin{aligned}
    \begin{cases} 
        \Delta u = f, & \text{in} \;\mathbb{R}_{+}^{3} \\ 
        u|_{\partial \mathbb{R}_{+}^{3}} = g   
        \end{cases} \\
    \Longrightarrow u(x) = \int_{R_{+}^{3}} G_{\mathbb{R}_{+}^{3}}(x-y)f(y)\,dy\\
    = \int_{\{ x_3>0 \}} \int_{R^{2}} \quad \text{"Horizontal Convolution"}
    \end{aligned}
\end{equation}
\begin{remark}
    This is not a convolution.
\end{remark}

\subsubsection{General props of Green's function}
\begin{equation}
    \begin{aligned}
        \begin{cases} 
            \Delta u = f, &  \\ 
            u|_{\partial \Omega} = 0, &   
            \end{cases} \\
    \iff u(x)= \int_{\Omega} G_{\Omega} (x,y) f(y)\,dy\\
    = \int_{\Omega} \delta_{x}(y)f(y)\,dy \\
    = f(x)
    \end{aligned}
\end{equation}

0 Boundary
\begin{align*}
    G(\cdot ,y)|_{\partial \Omega} = 0\\
    G(x,y) = 0 \quad \text{ when }x \in \partial \Omega
\end{align*}

Physically Green's function: Heat but the boundary keeps temperature 0.

Fix $y \in \Omega$. Look for sol to
\begin{equation}
    \begin{aligned}
    \Delta_{x}G_{\Omega}(x,y) = \delta(y-x) \\
    G_{\Omega}(\cdot ,y)|_{\partial \Omega} = 0\\
    G_{\Omega}(x,y) = K_0(x,y) + H^{(y)}(x)\quad \text{ where }K_0 = -\frac{1}{4\pi |x|} \text{ and }
        K_0(x,y) = -\frac{1}{4\pi|x-y|} \text{ where space Green's func}
    \end{aligned}
\end{equation}
where $H^{(y)}$ should be harmonic in $x$.

\begin{align*}
    \Delta_{x}(K_0(x,y) + H^{(y)}(x)) = \delta(y-x) + 0
\end{align*}

And 
\begin{align*}
    K_0(x,y) + H^{(y)}(x) = 0 \quad \text{ when }x \in \partial \Omega
\end{align*}

Then 
\begin{equation}
    \begin{cases} 
        \Delta_{x}H^{(y)}(x) = 0, &  \\ 
        H^{(y)}(x)|_{\partial \Omega} = K_0(x,y), &   
    \end{cases}
\end{equation}

Philosophy:
\begin{enumerate}
    \item  \sout{To solve the problem in $\Omega$, find $G_{\Omega}$.}
    \item [1*] Solve problem using a different method
    \item To find $G_{\Omega}$, need to solve problem for $H^{(y)}$.
    \item Read infrastion off of $G_{\Omega}$, like on HW4
\end{enumerate}

\begin{align*}
    G_{\Omega}(x,y) = G_{\Omega}(y,x)
\end{align*}
Green's function is symmetric 
\begin{equation}
    \begin{cases} 
    u,v \in C^{2}(\bar{\Omega}), &  \\ 
    u,v|_{\partial \Omega} = 0, &   
    \end{cases}
\end{equation}
\begin{align*}
    \int_{\Omega}\Delta u v = \int_{\Omega} u \delta v \\
    \langle Ax,y \rangle = \langle x, Ay \rangle \quad \forall \, x,y \in \mathbb{R}^{n}\\
    \iff A \text{ symmetric}
\end{align*}


\DATE{Sep 25, 2024}

\begin{equation}
    \begin{aligned}
    \Delta u =f \\
    u|_{\partial \Omega} = 0\\
    \iff \int_{\Omega}G(x,y)f(y)\,dy \quad \text{where you put the pole in }y\\
    \Delta_{x}G(x,y) = \delta(x-y)\\
    G(x,y) = 0 \quad x \in \partial \Omega
    \end{aligned}
\end{equation}

Question: 
\begin{equation}
    \begin{aligned}
    \Delta u = 0  \quad \text{in }\Omega\\
    u|_{\partial \Omega} = g
    \end{aligned}
\end{equation}

\begin{definition}[Green's formula]
    
\end{definition}

\begin{equation}
    \int_{\Omega}div \vec{v}\,dx = \int_{\partial \Omega}\vec{v} n \,dS
\end{equation}

\begin{align*}
    \vec{v} = \vec{u}f\\
    div(\vec{u}f) = (div\vec{u})f + \nabla f \cdot \vec{u}\\
    \int_{\Omega}(div \vec{u})f + \nabla f \cdot  \vec{u}\, dx = 
        \int_{\partial \Omega}(\vec{u}n)f\,dS\\
    \int_{\Omega}(div\vec{u})f \,dx\\
    = \int_{\partial \Omega}(\vec{u}n)f \,dS - \int_{\Omega}\nabla
\end{align*}

We have bounded terms
\begin{align*}
    \int(fy)' = \int f'g + fg'
\end{align*}

Suppose $u,v \in C^{2}(\bar{\Omega})$ scaler fns
\begin{align*}
    \int_{\Omega} \Delta u v \, dx = \int_{\partial \Omega}(\nabla u \vec{n})v \,dS 
    - \int_{\Omega} \nabla u \nabla v \, dx\\
    = \int_{\Omega}(\nabla u \vec{n})v \, dS - \int_{\Omega}u(\nabla v \vec{n})\, dS + 
    \int_{\Omega}u \Delta v \, dx
\end{align*}

Suppose u is harmonic, suppose $v(y) = G_{\Omega}(x,y)$. We substitute this into the above formula.

If $u$ is harmonic function in $\Omega$, then 
\begin{align*}
    u(x) = \int_{\Omega}u \frac{\partial G}{\partial n_{y}}(x,y)\, dS(y)
\end{align*}
This is the formula to solve 
\begin{equation}
    \begin{cases} 
    \Delta u = 0, &  \\ 
    u|_{\partial \Omega}= g, &   
    \end{cases}
\end{equation}

\begin{problem}
    Suppose define 
    \begin{align*}
        u(x) = \int_{\partial \Omega}g(y) \frac{\partial G}{\partial n_{y}}(x,y)\, dS 
    \end{align*}
    Then $u \longrightarrow g$ as $x \longrightarrow \partial \Omega$
\end{problem}

Here we call 
\begin{align*}
    P_{\Omega}(x,y) = \frac{\partial G}{\partial n_{y}}(x,y)
\end{align*}

\begin{align*}
    \frac{\partial }{\partial n} = - \frac{\partial }{\partial y_{3}} \\
    \frac{1}{4\pi}\frac{\partial }{\partial y_3}\Big(\frac{1}{2} 
    \frac{2(x_3 - y_3)}{|x_y|^{3}} + \frac{1}{2}\frac{2(x_3 + y_3)}{|x-y|^{3}} \Big)\\
    \frac{1}{2\pi} \frac{x_3}{|x-y|^{3}}
\end{align*}

\begin{align*}
    u(x) = \frac{1}{2\pi}\int_{\underbrace{\mathbb{R}^{3}}_{\partial \mathbb{R}_{+}^{3}}}
    \frac{x_3}{|x'-y'|^{2}+x_3^{2}}g(y)\,dy 
\end{align*}

\begin{align*}
    \frac{1}{2\pi} \frac{\epsilon}{(|y|^{2} + \epsilon^{2})^{\frac{3}{2}}} = \phi_{\epsilon}\\
    \text{(Check )} \phi = \frac{1}{2\pi} \frac{1}{(|y|^{2} + 1)^{\frac{3}{2}}}\\
    \phi_{\epsilon} = \frac{1}{2\pi} \frac{1}{\epsilon^{2}} \frac{1}{(|\frac{y}{\epsilon}|^{2})
    ^{\frac{3}{2}}}
\end{align*}
And we know $\int \phi \, dy = 1$

Conclusion:
\begin{align*}
    P((y,\epsilon), y) \text{ is a mollifier.}\\
    u(x_1,x_3,\epsilon) = \int_{\mathbb{R}^{3}} P(x_1-y_1, x_2-y_2, \epsilon)g(y_1,y_2)\,dy
\end{align*}

\begin{proposition}
    If $g$ is bdd and continuous, then $u \longrightarrow g$ as $x_3 \longrightarrow 0^{+}$ locally 
    uniformly.    
\end{proposition}

\begin{equation}
    \begin{aligned}
    \Delta u = 0  in \mathbb{R}_{+}^{3}\\
    u|_{\partial \mathbb{R}_{+}^{3}} = g
    \end{aligned}
\end{equation}

Question: uniqueness?

We want something like Liouville theorem in half space.

\begin{theorem}[comp principle]
    $u, v \in C^{2}(\bar{\Omega})$, if $u|_{\partial \Omega} = v|_{\partial \Omega}$, then 
    \begin{align*}
        u \equiv v
    \end{align*}
\end{theorem}

\begin{theorem}[Uniqueness]
    Suppose you have $u,v$ solve 
\begin{align*}
    \Delta u_{k} =f \text{in }\mathbb{R}_{+}^{3}\\
    u_{k}|_{\partial \mathbb{R}_{+}^{3}} = g, \quad k=1,2.
\end{align*}
Suppose $u,v \longrightarrow 0$ as $|x| \longrightarrow +\infty$, then
\begin{align*}
    u \equiv v 
\end{align*}
\end{theorem}

\DATE{Sep 27, 2024}

Heat equation:
\begin{align*}
    \partial_{s}u - \kappa \Delta u = 0 \quad \text{ where }u  = u(x,s)
\end{align*}
where $\kappa $ is the diffusivity

Change variable: $t = \kappa s$
\begin{align*}
    \partial_{t} &= \frac{1}{\kappa}\partial_{s} \\
    \Longrightarrow \partial_{t} v - \Delta v &= 0 \quad (\kappa = 1)
\end{align*}

Here we have $v(x,t) = u(x,s)$

Consider 
\begin{align*}
    \partial_{t}u - \Delta u = 0
\end{align*}
Fundamental solution? We consider two equivalent formulations:
\begin{align*}
    \partial_{t}\Gamma - \Delta \Gamma &= \delta_{(0,0)}(x,t) \\
    &= \delta_{0} \delta_0(t)
\end{align*}
The above is solved in $\mathbb{R}^{n} \times \mathbb{R}$


\begin{equation}
    \begin{cases} 
    \partial_{t}\Gamma - \Delta \Gamma = 0, &  \\ 
    \Gamma|_{t=0} = \delta_{0}(x), &   
    \end{cases}
\end{equation}
The above is solved in $\mathbb{R}^{n} \times (0,+\infty)$

Similarly, consider the ODE:
\begin{equation}
    \begin{aligned}
        \dot{x} = ax + \delta_0(t)& \\
        \text{or solve:}
    \begin{cases} 
    \dot{x} = ax,   \\ 
    x(0) = 1,    
    \end{cases}
    &\Longrightarrow x(t) = e^{at}
    \end{aligned}
\end{equation}

Symmetry of the heat equation:
\begin{enumerate}
    \item space, time translation 
    \item $O(n)$ spatial rotation and reflection 
    \item homogeneity $u \mapsto mu$ where $m \in \mathbb{R}$
    \item Scaling symmetry $u \mapsto u(\lambda x, \lambda^{2}t),\, \lambda > 0$ (Consider the former 
    setup: $[\kappa] = \frac{L^{2}}{\Gamma}$)
\end{enumerate}

What's the meaning of a function which is homogeneous?
\begin{align*}
    f(\lambda x) = \lambda^{\alpha}f(x) ,\quad \lambda>0\\
    \iff f \text{ is }\alpha-\text{homogeneous}
\end{align*}

\begin{example}
    $\frac{1}{|x|}$ is $(-1)$-homogeneous.
    \begin{align*}
        \frac{1}{\lambda |x|} = \frac{1}{\lambda} \frac{1}{|x|} = \lambda^{-1} \frac{1}{|x|}
    \end{align*}
\end{example}

Here for $\delta_0(x)$:
\begin{align*}
    \delta_{0}(x) &= \lim_{\epsilon \to \infty} \phi_{\epsilon}(x) \\
    &= \lim_{\epsilon \to o^{+}} \frac{1}{\epsilon^{n}}\phi(\frac{x}{\epsilon})
\end{align*}

Homogeneity of $\delta_0(x)$:
\begin{align*}
    \delta_0(\lambda x) &= \lim_{\epsilon \to 0^{+}} \phi_{\epsilon}(\lambda x)\\
     &= \lim_{\epsilon \to 0^{+}} \frac{1}{\epsilon^{n}}\phi(\frac{\lambda x }{\epsilon})\\
     &= \lim_{\frac{\epsilon}{\lambda} \to 0^{+}}\frac{1}{\lambda^{n}}(\frac{\lambda}{\epsilon})^{n}
        \phi(\frac{\lambda x }{\epsilon}) \\
    &= \lambda^{-n}\delta_0(x)
\end{align*}
This shows that $\delta_0$ is $(-n)$-homogeneous.

Then with the scalling symmetry, we have:
\begin{align*}
    &u_{\lambda,m}(x,t) = mu (\lambda x, \lambda^{2}t)\\
    &u_{\lambda}(x,t) := \lambda^{m}u(\lambda x, \lambda^{2}t)\\
    (m = \lambda^{n}) &\quad f_{\lambda}(x) := \lambda^{n}f(\lambda x)\\
    &f_{\lambda} = f \iff (-n)-\text{homogeneous}
\end{align*}

Observation:
\begin{align*}
    (\delta_0)_{\lambda} = \delta_0(x)
\end{align*}

Expect: $\Gamma_{\lambda} = \Gamma$

In Evans:
\begin{align*}
    \Gamma = \frac{1}{t^{\alpha}}F(\frac{x}{t^{\beta}})
\end{align*}
It seems we "guess" this form for heat equation. But by the derivation we've done above for 
scaling symmetry, we know this must be the form.

By symmetry, we know that:
\begin{align*}
    \Gamma(x,t) &= \lambda^{n} \Gamma(\lambda x, \lambda^{2} t), \quad \forall \lambda > 0\\
    (\text{Choose } &\lambda = \frac{1}{t^{\frac{1}{2}}})\\
    \Gamma(x,t) &= \frac{1}{t^{\frac{n}{2}}}\Gamma(\frac{x}{t^{\frac{1}{2}}}, 1)\\
    &= \frac{1}{t^{\frac{n}{2}}}F(\frac{x}{t^{\frac{1}{2}}}) \\
    F(y) &= \Gamma(y,1)
\end{align*}
Here we define: $y := \frac{x}{t^{\frac{1}{2}}}$

\begin{equation}
    \begin{aligned}
        \partial_{t}\Gamma - \Delta \Gamma = 0 \\
        \partial_{t}(t^{-\frac{n}{2}} F(\frac{x}{t^{\frac{1}{2}}}))
        - \Delta(t^{-\frac{n}{2}}F(\frac{x}{t^{\frac{1}{2}}})) = 0
    \end{aligned}
\end{equation}

Differentiate it:
\begin{align*}
    &\partial_{t}(t^{-\frac{n}{2}} F(\frac{x}{t^{\frac{1}{2}}}))
        - \Delta(t^{-\frac{n}{2}}F(\frac{x}{t^{\frac{1}{2}}})) 
    = -\frac{n}{2}t^{-\frac{n}{2} - 1}F + t^{-\frac{n}{2}}(\nabla_{y}F)(-\frac{1}{2}xt^{-\frac{3}{2}})
    - t^{-\frac{n}{2}}(\Delta_{y}F)t^{-1} = 0\\
    &\Longrightarrow -\frac{n}{2}F - \frac{1}{2}y \nabla_{y}F - \Delta_{y}F = 0
\end{align*}

Here we know F radial: $F = F(r),\, r= |y|$, $y \nabla_{y} = r\partial_{r}$

Solve the ODE:
\begin{align*}
    \frac{n}{2}F +\frac{1}{2}r \partial_{r}F + \partial_{r}^{2}F + \frac{n-1}{r}\partial_{r}F &= 0\\
    \frac{1}{2r^{n-1}}(r^{n}F)' + \frac{1}{r^{n-1}}(r^{n-1}F')' &= 0\\
    \frac{1}{2}r^{n}F + r^{n-1}F' = A &= 0\\
    F' &= -\frac{1}{2}rF\\
    F &= Be^{\frac{-r^{2}}{4}}
\end{align*}

What is B?
\begin{align*}
    \partial_{t}\int \Gamma - \int\Delta\Gamma &= 0\\
    \frac{d}{dt}\int_{\mathbb{R}^{n}}\Gamma &= 0 \\
    \Longrightarrow \int\Gamma \,dx &= \text{ constant in time}
\end{align*}

\begin{align*}
    \int_{\mathbb{R}^{n}}F(|y|)\,dy = 1 \Longrightarrow   B = \frac{1}{(4\pi)^{\frac{n}{2}}}
\end{align*}

The Heat Kernel:
\begin{align*}
    \Gamma(x,t) = \frac{1}{(4\pi t)^{\frac{n}{2}}}e^{-\frac{|x|^{2}}{4t}}
\end{align*}

\begin{align*}
    \Gamma(x,\epsilon^{2}) = \phi_{\epsilon} = \frac{1}{\epsilon^{n}}\phi(\frac{x}{\epsilon})
    \; \text{ is a Mollifier.}
\end{align*}

\begin{theorem}
    $u_0$ is bounded and continuous on $\mathbb{R}^{n}$. Define 
    \begin{align*}
        u(x,t) &:= (\Gamma(\cdot , t) * u_0)(x_0) \\
        &= \int_{\mathbb{R}^{n}}\Gamma(x-y, t)u_0(y)\,dy 
    \end{align*}
    , which solves heat equation and 
    \begin{align*}
        u \longrightarrow u_0 \text{ as  }t \longrightarrow 0^{+} \quad \text{locally uniformly.}
    \end{align*}
\end{theorem}


\DATE{Oct 2, 2024}

Heat Kernel:
\begin{align*}
    \Gamma (x,t) = \frac{1}{(4\pi t)^{\frac{n}{2}}} \exp(-\frac{|x|^{2}}{4t})
\end{align*}

Solve: IVP
\begin{equation}
    \begin{cases} 
    \partial_{t} u - \Delta u = 0, &  \\ 
    u| = u_t, &   \\
    t = 0
    \end{cases}
\end{equation}
$u(x,t) = \Gamma(,t)\ast u_{0} = \int _{\mathbb{R}^{n}}\Gamma(x-y,t)u_0(y)dy$

\begin{equation}
    \begin{cases} 
    \partial_{t}u - \Delta u = f, &  \\ 
    u| = 0, &   \\
    t = 0
    \end{cases}
\end{equation}

$f: \mathbb{R}^{n} \times [0,T) \to \mathbb{R}$ extend by zero.

We have 
\begin{align*}
    \Gamma \ast_{x,t}f = \int_{-\infty}^{\infty}\int_{\mathbb{R}^{n}}\Gamma(x-y,t-s)
    f(y,s)  dy \mathrm{d}s
\end{align*}
We only care when $t \ge s$. And we know $s \ge 0$.

Thus we have 
\begin{align*}
    \Longrightarrow = \int_{s=0}^{t}\int_{\mathbb{R}^{n}}\Gamma(x-y, t-s)f(y,s)ds
\end{align*}

Define 
\begin{equation}
    \Gamma_{1}
    \begin{cases} 
    \Gamma, &|x|^{2} + t  \ge 1 \\ 
    smooth, &|x|^{2} + t \le 1    
    \end{cases}
\end{equation}
Define $\Gamma_{\epsilon}(x,t) = \frac{1}{\epsilon^{n}}
\Gamma(\frac{x}{\epsilon}, \frac{t}{\epsilon^{2}})$

Consider $(\partial_{t} -\Delta)\Gamma_{\epsilon} = \phi_{\epsilon} = \frac{1}{\epsilon}
\phi(\frac{x}{\epsilon}, \frac{t}{\epsilon^{2}})$


WTS: $\Delta(G \ast f) = f$
\begin{align*}
    \Delta(G_{\epsilon} \ast f)
\end{align*}

\begin{proposition}
    Suppose $f \in C^{2}_{1}(\mathbb{R}^{n} \times [0,T))$ and $f$ is compactly supported in 
    $\mathbb{R}^{n} \times [0,T)$
\end{proposition}
\begin{remark}
    Notice we let the function touches the initial time.
\end{remark}


\begin{equation}
    \begin{aligned}
        u = \Gamma \ast_{x,t}f \\
        u \in C_{1}^{2}\text{ solves}\\
        \begin{cases} 
        \partial_{t}u - \Delta u = f, &  \\ 
        u| = 0, &   \\
        t = 0
        \end{cases}
    \end{aligned}
\end{equation}

\begin{remark}
    Suppose we have $u \in L^{1}_{loc}(\mathbb{R}^{n} \times [0, T))$, then we can say:
    \begin{align*}
        \partial_{t}u - \Delta u = f \text{ weakly in }\mathbb{R}^{n}\times[0,T)
    \end{align*}
    If $\int u (-\partial_{t} - \Delta)\varphi = \int f \varphi \quad \forall \varphi \in 
    C^{\infty}_{0}(\mathbb{R}^{n} \times (0,T))$ (don't see initial time).
    \begin{align*}
        f \in L^{1}_{x,t}(\mathbb{R}^{n} \times (0,T))\\
        \Longrightarrow u \text{ solves equation weakly.}
    \end{align*}
\end{remark}

Now think about analogy of harmonic function: $\Delta u = 0$.
We call them caloric function satisfying: $\partial_{t}u - \Delta u = 0$.

\begin{enumerate}
    \item Interior estimates
    \item Liouville theorem 
    \item maximum principle 
    \item Harneck
    \item Weyl's lemma.
\end{enumerate}

Now let's talk about interior estimates: if $u \in C^{2}_{1} \& \text{compactly supported}$, then 
\begin{align*}
    \Gamma \ast_{x,t}(\partial_{t} - \Delta)u = u
\end{align*}
Now we define $B_{r} := \{ x\in \mathbb{R}^{n}: |x| < r \}$ and $Q_{r} : =
B_{r} \times (-r^{2}, 0)$.

If $u$ is caloric function on $Q_{2R}$, then
\begin{equation}
    \sup_{Q_{R}}|\partial_{t}\nabla_{x}^{k}u| \le C_{j,k} \frac{1}{R^{2j+k}} 
    \barint_{Q_{2R}\setminus Q_{R}}|u| dxdt \quad j,k \ge 0
\end{equation}
We think about $u = \phi u
= \Gamma \ast (\partial_{t} - \Delta)(\phi u) = \Gamma \ast 
(\partial_{t}\phi u - \Delta \phi u - 2\nabla \phi \nabla u)$ 
where $\phi \equiv 1$ on $Q_{R}$ and $\phi \equiv 0$ outside $Q_{2R}$.


\begin{theorem}[Liouville]
    If $u$ is bounded and ancient sol to 
    $\partial_{t}u - \Delta u = 0$ on $\mathbb{R}^{n} \times (-\infty,0)$, then 
    $u \equiv \operatorname{const}$
\end{theorem}
\begin{remark}
    We know 
    \begin{align*}
        \le C_{j,k} \frac{1}{R^{2j+k}} 
    \barint_{Q_{2R}\setminus Q_{R}}|u| \le C \frac{1}{R^{2}}\sup_{\mathbb{R}^{n} \times (-\infty,0)}
    |u|
    \end{align*}
\end{remark}

Weyl's lemma: weakly caloric function is smooth.

\begin{theorem}[Maximum principle]
    $\Omega$ is bounded domain $\subseteq \mathbb{R}^{n}$. Suppose $u \in C^{2}_{1}(\Omega \times (0,T))
    \bigcap C(\overline{\Omega \times [0,T]})$. (This is kind of a cylinder).

     $\partial_{par}(\Omega \times (0,T)) = \underbrace{(\partial \Omega \times [0,T] \times 
     (\bar{\Omega} \times \{ 0 \}))}_{\text{sides and bottom}}$

     \begin{align*}
        \text{False:} \max_{\Omega \times [0,T]} > \max_{\partial_{par}(\Omega \times [0,T])}u
     \end{align*}
     If $u(x_0,t_0) = \max_{\Omega \times [0,T]}u$ and $(x_0,t_0) \notin \partial_{par}
     (\Omega \times [0,T])$, then 
     \begin{align*}
        u \equiv \operatorname{const}
     \end{align*}
\end{theorem}

\DATE{Oct 4, 2024}
Parabolic max principle: understanding

Suppose $u$ has local max at $x_0, t_0$
\begin{enumerate}
    \item In the domain $\Omega$. $\partial_{t}u(x_0, t_0) = 0$ and $\nabla u(x_0,t_0) = 0$
    We have 
    \begin{align*}
        u(x,t) = u(x_0,t_0) + \frac{1}{2}(\nabla^{2}u(x_0,t_0)(x-x_0,t-t_0))(x-x_0,t-t_0)
    \end{align*}
    And we have 
    \begin{align*}
        \underbrace{\partial_{t}u}_{ = 0} - \Delta u(x_0,t_0) = 0
    \end{align*}
    We actually have 
    \begin{align*}
        -\Delta u (x_0,t_0) < 0
    \end{align*}
    We will add $\epsilon(t + |x|^{2})$.
    \item $\partial_{t}u(x_0,t_0) \ge 0$ and we still have $\nabla_{x}u(x_0,t_0)$.
    Consider heat function
    \begin{align*}
        0 = \partial_{t}u - \Delta u(x_0,t_0) \ge -\Delta u(x_0,t_0)
    \end{align*}
    It's a parabola or saddle.

\end{enumerate}

\begin{theorem}[Uniqueness]
    (We prove uniqueness by maximum principle) $u \in C_{1}^{2}(\mathbb{R}^{n}\times (0,T))$ and 
    $u \in C(\overline{\mathbb{R}^{n} \times [0,T]})$, $u|_{t=0} = 0$ and 
    $|u| \longrightarrow 0$ as $|x| \longrightarrow \infty$.

    Then $u \equiv 0$.
\end{theorem}

We prove uniquesness by another way 
\begin{align*}
    \dot{x} = Ax
\end{align*}
where $x \in R^{10}, A \in R^{10 \times 10}$
\begin{equation}
    \begin{cases} 
    u = x, & \in L^{2}  \\ 
    \Delta = A, &   
    \end{cases}
\end{equation}
By this example, we can try Parabolic PDF $\iff$ ODEs in $\infty$-dim


\begin{equation}
    \begin{cases} 
    \dot{x} = Ax, &  \\ 
    x(0) = \vec{x}, &   
    \end{cases}
\end{equation}
defined on $(0,T)$. We solve it and obtain $x(t) = e^{At}\vec{0} = \vec{0}$

We solve 
\begin{equation}
    \begin{cases} 
    -\dot{y} = A^{*}y + f(t), & \text{on } (0,T)  \\ 
    y(T) = 0, &
    \end{cases}
\end{equation}

We have this "final time" problem $\longrightarrow$ solve backward.

Suppose $\forall  f$ solve adj problem existence.

Existence for adj problem $\Longrightarrow$ uniqueness for forward problem.

Given $x$. Let $f$ be arbitrary $(f \in C_{0}^{\infty}((0,T)))$.

\begin{equation}
    \begin{cases} 
    \dot{x} = Ax, &  \\ 
    \int_{0}^{T}\langle \frac{dx}{dt}, y \rangle = \int_{0}^{T}\langle Ax,y \rangle,   
    \end{cases}
\end{equation}

\begin{align*}
    -\int_{0}^{T}\langle x,\dot{y} \rangle = \int_{0}^{T}\langle x, A^{*}y \rangle\\
    \Longrightarrow \int_{0}^{T}\langle x, \dot{y} + A^{*}y \rangle \,dt = 0 \\
    \int_{0}^{T}\langle x(t), -f(t) \rangle dt = 0
\end{align*}
This is true for $\forall  f$.
Then $\Longrightarrow \, x \equiv 0$.

\begin{theorem}
    $u \in C_{1}^{2}(\mathbb{R}^{n} \times (0,T)) \bigcap C(\overline{\mathbb{R}^{n} \times [0,T]})$.
    $u|_{t=0} = 0$, $|u| \le Ce^{A|x|^{2}}$.

    Then $u \equiv 0$.
\end{theorem}

We are solving 
\begin{align*}
    v(x,t)\cdots\\
\end{align*}

Let $f \in C_{0}^{\infty}(\mathbb{R}^{n} \times (0,T))$
\begin{equation}
    \begin{cases} 
        -\partial_{t}v = \Delta v + f, & \text{on }\mathbb{R}^{n} \times (0,T)  \\ 
    v|_{t=T} = 0  
    \end{cases}
\end{equation}

Let $w(x,t) = v(x,T-t)$ and $\tilde{f}(x,t) = f(x,T-t)$ where $w(x,t) = \Gamma *_{x,t} \tilde{f}$

\begin{align*}
    w(x,t ) = \int_{0}^{t}\int_{\mathbb{R}^{n}}\Gamma(x-y, t-s) \tilde{f}(y,s) \,dyds \\
    \le \int_{0}^{t}\int_{B_{R}}e^{\frac{-|x-y|^{2}}{4(t-s)}}|\tilde{f}(y,s)|
\end{align*}

existence $\checkmark$.

\begin{align*}
    |w| \le \tilde{C}e^{\frac{-|x|^{2}}{Ct}}
\end{align*}

Consider 
\begin{align*}
    \partial_{t}u - \Delta u = 0\\
    \int_{0}^{T}\int \partial_{t} u \cdot v - \int_{0}^{T}\int \Delta u v = 0\\
    \Longrightarrow \underbrace{\int_{\mathbb{R}^{n}}u(x,T)v(x,T)}_{= 0} - \int_{\mathbb{R}^{n}}
        u(0)v(0) - \int \int u(\partial_{t}v + \Delta v) = 0\\
    \Longrightarrow \int \int u(-f)\,dx dt = 0
\end{align*}
This is true $\forall f \in C_{0}^{\infty}(\mathbb{R}^{n})\times (0,T) \Longrightarrow u \equiv 0$.


\end{document}